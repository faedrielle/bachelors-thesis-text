% arara: xelatex
% arara: xelatex
% arara: xelatex


% options:
% thesis=B bachelor's thesis
% thesis=M master's thesis
% czech thesis in Czech language
% english thesis in English language
% hidelinks remove colour boxes around hyperlinks

\documentclass[thesis=B,english]{FITthesis}[2019/03/21]

\usepackage{todonotes}
\usepackage{graphicx}
\usepackage{setspace} % for \onehalfspacing and \singlespacing macros


\usepackage{etoolbox}
\AtBeginEnvironment{quote}{\singlespacing\small}

\usepackage{enumitem}

\setlength{\fboxsep}{0.005pt}
\newcommand{\tmpframe}[1]{\fbox{#1}}
%\renewcommand{\tmpframe}[1]{#1}

\def\code#1{\texttt{#1}}

%\usepackage[utf8]{inputenc} % LaTeX source encoded as UTF-8
% \usepackage[latin2]{inputenc} % LaTeX source encoded as ISO-8859-2
% \usepackage[cp1250]{inputenc} % LaTeX source encoded as Windows-1250

% \usepackage{subfig} %subfigures
% \usepackage{amsmath} %advanced maths
% \usepackage{amssymb} %additional math symbols

\usepackage{dirtree} %directory tree visualisation

% % list of acronyms
% \usepackage[acronym,nonumberlist,toc,numberedsection=autolabel]{glossaries}
% \iflanguage{czech}{\renewcommand*{\acronymname}{Seznam pou{\v z}it{\' y}ch zkratek}}{}
% \makeglossaries

% % % % % % % % % % % % % % % % % % % % % % % % % % % % % % 
% EDIT THIS
% % % % % % % % % % % % % % % % % % % % % % % % % % % % % % 

\department{Department of Software Engineering}
\title{Analysis of an IoT solution for assessment of physical difficulty of tourist tracks using smart wearables}
\authorGN{Zuzana} %author's given name/names
\authorFN{Václaviková} %author's surname
\author{Zuzana Václaviková} %author's name without academic degrees
\authorWithDegrees{Zuzana Václaviková} %author's name with academic degrees
\supervisor{Ing. Jan Krákora, PhD.}
\acknowledgements{I would like to thank my supervisor, Ing. Jan Krákora, PhD., for his guidance and encouraging words when I was out of my depth. Many thanks to \{v S}imon Let for reviewing my work throughout the writing process and giving valuable advice. Finally, thank you to my partner David for providing endless support, and my family and friends, for never ceasing to believe in me.}
\abstractEN{The subject of this bachelor's thesis is to perform a software analysis of an Internet of Things platform that estimates how difficult tourist tracks will be for people before they take them, based on heart rate data collected from their and other users' smartwatches.

The research chapters of the thesis focus on modern methods of cardiovascular fitness assessment and the study of existing applications in the domain of fitness and e-health.
Using methods of agile development, the functionality of the solution is analysed in the form of user stories, which can be directly taken as assignments for development teams.
This analysis is the foundation for designing the front-end of a mobile application that supports navigation, informed planning of hikes both alone and with other users based on all the participants' fitness, and smartwatch integration. The whole design is built with an emphasis on user experience guidelines.

Subsequently, a high-level architecture of the solution is designed, and a proof of concept of the platform is developed, however, due to technical difficulties, it is unfinished.}
\abstractCS{Predmetom bakalárskej práce je softvérová analýza platformy Internetu vec{\' i}, ktorá odhaduje náročnosť turistických trás pre turistov na základe údajov o srdcovej činnosti zozbieraných zo zariadení smartwatch patriacich používateľom.

Výskumné kapitoly práce sa sústredia na moderné metódy pre hodnotenie kardiovaskulárnej kondície, ako aj štúdiu existujúcich aplikácií v oblasti fitness a e-health.
V práci je využitá metodika agilného vývoja. Funkcionalita riešenia je analyzovaná fo forme user stories, ktoré môžu byť priamo preberané ako zadania pre vývojárske tímy. Táto analýza je základom pre návrh front-endovej časti mobilnej aplikácie, ktorá podporuje navigáciu, integráciu zariadení smartwatch a informované plánovanie túr samostatne alebo spolu s inými používateľmi aplikácie na základe ich fyzickej kondície. Celý design je vytvorený s dôrazom na smernice pre user experience.

Okrem toho je nadizajnovaná high-level architektúra riešenia a vyvinutý proof of concept platformy, ktorý je z dôvodu technických problémov nedokončený.}
\placeForDeclarationOfAuthenticity{Prague}
\keywordsEN{Internet of Things, hiking track difficulty, heart rate monitoring, physical fitness assessment, smart wearables, software analysis, mobile application design, user experience}
\keywordsCS{Internet vecí, náročnosť turistickej trasy, meranie tepu srdca, posudzovanie telesnej kondície, nositeľná elektronika, softvérová analýza, dizajn mobilnej aplikácie, užívateľské skúsenosti}
\declarationOfAuthenticityOption{1} %select as appropriate, according to the desired license (integer 1-6)
% \website{http://site.example/thesis} %optional thesis URL

\graphicspath{ {./Images/} }

\begin{document}

% \newacronym{CVUT}{{\v C}VUT}{{\v C}esk{\' e} vysok{\' e} u{\v c}en{\' i} technick{\' e} v Praze}
% \newacronym{FIT}{FIT}{Fakulta informa{\v c}n{\' i}ch technologi{\' i}}


\setsecnumdepth{part}
\chapter{Task - temporary chapter}
\linebreak
Analysis of an IoT solution for assessment of tourist track physical difficulty using smart wearables

The aim of this thesis is to analyse, design and implement a PoC IoT solution for tourist track difficulty estimation.\todo[color=green]{add a dictionary of shortcuts - slovník pojmov}
Using a mobile application, the user shall be able to select a track and assess how much effort the specific part of the track would require from them, based on data collected from similarly fit users.
\todo[color=green]{reword into sentences, no points}
\begin{enumerate}
    \item Analyse and compare similar existing solutions.
    \item Analyse and design an IoT solution for data processing.
    \item Design a mobile application for data visualization and track selection.
    \item Create a PoC of the IoT platform and of the mobile application.
    \item Demonstrate the solution's functionality on a relevant use case, using real data collected from smart wearables.
\end{enumerate}


\setsecnumdepth{all}
\chapter{Introduction}
\linebreak
With the growing popularity of smart wearables in the general public, a number of groups of people have been able to adjust their behaviour according to the way their data gets processed and fed back to them.
People have been saving up for the Apple Watch 5 \cite{AppleWatch5} in spite of the ECG health monitor being virtually unnecessary for those not at risk of cardiac disease \cite{ecg-screening},
runners can barely imagine not checking how fast and how efficiently they just ran their daily track or how much they have improved over the last month,
people with sedentary jobs attempt to keep their daily step count above a currently recommended number, etc.
All of this data is collected from a variety of devices, such as smart watches, textiles with integrated sensors, or implants, all of which are meant to be worn on the user's body and which measure and send data for analysis - hence the name \textit{wearable technology}, or \textit{smartwear}.\todo[color=green]{get a source}

The result of this thesis will be a design for an IoT solution, mainly to be used by hikers who want to not only see in advance how long a track is going to take them (which generally isn't tailored to the individual's fitness),
but also an objective difficulty level of particular sections of said track, which may have additional benefits for people with blood pressure issues, who are interested in a more detailed insight into the heart rate induced by the track\todo[color=green]{wording}.

Using methods of software engineering, I am going to analyse the use of wearable technology in existing applications and document the most common features as well as their outstanding traits and their use of IoT principles (or lack thereof).
Based on this analysis, I will pick the most relevant features to include in my design and extend them with features of my own devising, which will set my solution apart from existing software.\todo[color=green]{is this sentence necessary?}

I will design and implement a PoC IoT solution - a mobile application which will collect data from the smart watch sensor and from the smart phone's GPS locator, a backend IoT platform for receiving and processing of collected data, and the communication channels used between these two modules.
I will neither design nor implement a full-fledged IoT platform - that's not in the scope of this thesis.
I will design (not implement) the frontend mobile application for displaying of processed data, track selection and explore the necessary legal requirements for the use of integrated maps \todo[color = green]{introduce the maps in some other way?}.

I will study the protocols used by a smart watch \todo[color=green]{use specific name once chosen} and adjust my mobile application accordingly.

I will also perform a data analysis and propose suitable operations to be executed on the collected data, which will be included in the processing of this data on the IoT platform.
\todo[color=green]{reword to not always use 'I will'}



\chapter{Aims}
The aim of this thesis is to analyse and design the functionality of an IoT solution for assessment of tourist track difficulty, based on the individual's fitness.

The early chapters of this thesis will be the analysis of the ways fitness is currently assessed and their relevance to my application,
followed by a study of existing applications in the domain of fitness trackers and e-health, and an assessment of how important specific features are to the applications' main aims.
I am going to apply the results of this research to my own approach when determining the required functionality for the system.

Based on these descriptions, I am going to design the front-end of a mobile application for track selection and data visualization.
I am also going to design a high-level architecture of the IoT solution.

In the last part of the thesis I am going to focus on implementing a PoC of the designed solution and processing of collected data.
\chapter{Fitness assessment}
 physical fitness test usually evaluates multiple components of one's health, including cardiovascular endurance, muscular strength, muscular endurance, flexibility and body composition.

In this chapter I will focus on familiarizing the reader with cardiovascular endurance testing, as it has the largest effect on hiking.
There is a number of indicators that one can measure (such as the lactate threshold - a non-linear increase in blood lactate concentration\cite{lactate-threshold}), 
but the most popular and easiest to measure with a heart rate monitor is VO2 max.

\section{VO2 and VO2 max}

The volume of consumed oxygen (VO2) is closely correlated with heart rate, respiration rate, and on/off-dynamics, all of which can be derived from RR interval data (time elapsed between two sucessive R waves -- the tallest spikes on an ECG),
with respiration rate being able to differentiate between metabolic (physical acitivity induced) and non-metabolic (mental and non-exercise related physical stress) changes in heart rate,
which makes models that take it into account highly accurate compared to those which only use heart rate.\cite{vo2-hr-firstbeat}

VO2 max, the maximal value of VO2, is considered the most accurate metric of cardiovascular fitness.

Measured in millilitres of oxygen consumed per minute or millilitres of oxygen per kilogram of bodyweight per minute, it is the amount of oxygen the body is able to effectively use (transform to energy) during intense or maximal exercise.
With the growing amount of oxygen the body can consume, it can perform better during strenuous activity and give improved results.\cite{vo2max-definition}
This is limited by the ability of the cardiorespiratory system to deliver oxygen to the exercising muscle, thus making it impossible for an athlete to operate above 100\% of their VO2 max for extended periods.\cite{vo2max-oxygen-delivery}

\begin{figure}[h]
    \includegraphics[width=\textwidth]{V02max-running.png}
    \caption{With increasing speed of the runner, oxygen consumption increases linearly and plateaus around 18.5 kilometres per hour - the VO2 max. \cite{vo2max-speed-img}}
\end{figure}

\subsection*{How to measure Vo2 max}

\subsubsection*{Laboratory test}

Typically, VO2 max is measured directly in laboratory conditions while wearing a respiratory mask, by analyzing inspired and expired breathing gases during maximal exertion,\cite{vo2max-definition} usually running on a treadmill or riding a stationary bike.

This method of determining VO2 max is highly accurate, but given the need for expensive equipment and trained staff, laboratory testing just isn't feasible for everyday population-wide testing.
On the other hand, results from laboratory tests are often used as reference values for determining the accuracy of alternative methods.

\subsubsection*{Cooper test - Twelve-Minute Run-Walk}

First suggested in the 1970's, Cooper's Twelve-Minute Run-Walk is an endurance test, where the main goal is to run (or walk) as long a distance as possible in twelve minutes.
It was originally developed mainly for armies and police agencies, but it's also popularly (and unnecessarily\cite{cooper-pupils}) used in schools on untrained pupils.
It is inaccurate for people who do not train running and swim or bike instead, as their bodies are used to a completely different set of movements.

There is a high correlation between the distance an individual can run and their VO2 max value, which can be calculated thusly:

$VO2max = (22.351 \times distance\_in\_kilometers) - 11.288$\cite{cooper-vo2max}

\subsubsection*{Multistage Fitness Test - Beep test}

Introduced by a Canadian sport scientist Luc Léger in the 1970's, this aerobic test has consistently been found a reliable way of finding a person's VO2 max.

The original version of the beep test (the Track Test) had the participants running back and forth in the interval of two minutes while the running pace was increased, so they always had to run further than in the previous iteration.

This test was highly efficient, however, given the long intervals and spatial requirements, it couldn't be performed indoors, giving rise to modifications, such as the Twenty-Metre Shuttle Run Test.
As the name suggests, two markers are placed twenty metres apart and again, the participants run back and forth between them, trying to reach the opposite marker before the next -- shorter -- stage begins, that is, before the next beep.
Once the test subject fails to reach a marker, turns without touching the marked line, or starts running before the signal, after one warning, their test is over.
They can also choose to stop when they have reached their maximum physical limit.

The test is divided into stages with recommended running speeds, each stage consisting of multiple shuttles - see the table at \cite{beep-test-scoring-table}.
From the stage and shuttle reached, one can calculate the VO2 max using the formula

\[VO2max (\frac{ml}{kg*min-1}) = 31.025 + 3.238X - 3.248A + 0.1536AX\]

where $A$ is age of the participant and $X$ is the speed in the final shuttle.
This formula fits the commonly used version of the Beep test (starting at 8.0 km/h, jumping to 9.0 km/h and continuing to rise by 0.5 km/h per level) and may vary depending on the version used.\cite{beep-test-versions}\cite{beep-test-20m-valid}

This modified version (just like the original) has been recognized as a valid method to determine VO2 max of male and female adults, individually or in groups, on most gymnasium surfaces.\cite{beep-test-20m-valid}

In 2017, Tomkinson et al. published a complex systematic study of nearly 1.2 million 9-17 years old children from 50 countries with data from beep tests as old as 1981,
setting standardized norms for testing of fitness in the world's youth.\cite{beep-test-youth-large-study}


\subsubsection*{Six-Minute Walk Test}

There are also some non-running tests which can be used to measure an individual's VO2 max.

The Six-Minute Walk Test, or 6MWT for short, originated from the Cooper test in the 1980's as a more feasible and less exhausting alternative.\cite{6mwt-history}

The main objective is for the participant to walk as long a distance as they can in the provided six minutes on a 15- or 30-metre long track, making sharp turns at the end.
Mänttäri et al. published a study in 2018\cite{6min-walk-test-mantarri}, aimed to develop a prediction model for VO2 max based on 6MWT results combined with heart rate at the end of the 6MWT and easy-to-measure anthropometric and demographic data (such as weight, height, age and gender).
The different track lengths only showed negligible differences between distances walked (on average, 23.5 m longer on the 30-m course).
\textit{For men, the best predictors for VO2max were walking distance, age, BMI, heart rate at the end of 6MWT and height, and for women, walking distance, age and weight} -- surprisingly, heart rate wasn't a significant factor for the female participants.

The formula they developed for men:

 $VO2 max=110.546 + 0.062\times distance - 0.25\times age - 0.486\times BMI - 0.42\times height - 0.109\times HR$

 and for women:

 $VO2 max=22.506 - 0.271\times weight + 0.051\times distance - 0.065\times age$

 \textit{where $weight$ is body weight (kg), $distance$ is distance walked in 6 min (m), $age$ (years), $BMI$ is calculated body mass index (kg/m2), $height$ is body height (cm) and $HR$ is heart rate at the end of the walking test (bpm).}

The authors themselves conclude that their output should be cross-validated with new independent samples of adult population,
but they consider it a precise enough method to be used for public health monitoring in adults.

Another study from 2015\cite{6min-walk-test-burr} instead used resting heart rate and arrived at this formula:

$VO2 max =70.161 + 0.023 \times distance - 0.276 \times weight - 6.79 \times sex - 0.193 \times resting_HR - 0.191 \times age$

using $distance$ walked during 6MWT (m), $weight$ (kg), $sex$ -- 1 for female, 0 for male, and $resting_HR$ for resting heart rate in beats per minute.

\subsubsection*{Sit-to-Stand test}

Another test that doesn't require much in terms of space or equipment and is in fact a common daily activity, which is often used to test lower body strength and endurance in older adults, has been studied for correlation with VO2 max values.
The Sit-to-Stand test, which consists of repetitive sitting down and standing up, has acquired a number of different forms just like most other tests used by the general public.
The versions include but aren't limited to the incremental twelve minute STS test -- sit and stand in incrementally shorter periods of time, repeated for twelve minutes, often controlled by a metronome \cite{seat-height-sit-to-stand};

Having been closely studied by Nakamura et al., it is a potentially valid test (when done with arm support, to increase the complete exhaustion threshold, and the right chair height, among other factors), but more research needs to be carried out to find a suitable version.\cite{frequencies-sit-to-stand}\cite{validity-sit-to-stand}

\subsubsection*{Non-exercise tests}

There are also some tests that do not require exercise and instead, have the individual fill out a questionnaire, such as the Jackson test\cite{nonexercise-vo2max-test-jackson} or the George test\cite{nonexercise-vo2max-test-george},
which ask questions on the topic of an individual's exercise habits, such as their perceived level of activity over the last six months.
Every answer has a specific weight in the resulting equation, based on which the person's fitness is evaluated.

These tests, however, should not be used as a sole source of information especially in medicine, as the data is self-reported and may be influenced by bias or conscious falsehood.

\section{The Performance Condition}

A person's VO2 max is recognized as their baseline fitness level index, however, everybody has days when they perform better, and days when they perform worse, while the factors in play are the amount of sleep they get, being sore from previous workouts, and general physical and mental wellbeing.

That is why Firstbeat introduced the Performance Condition.

Firstbeat is a provider of physiological analytics for sports and well-being who have translated human physiology into mathematical models based on heart rate variability (see below).
They provide their analytics engine as a service to a number of fitness device manufacturers, such as Garmin, Xiaomi, Honor and others.

The Performance Condition is a real-time index of a user's immediate fitness and fatigue level compared to their baseline fitness level.

This indicator is measured on a scale of -20 to +20, with each point representing roughly 1\% of their Vo2 max.
So if the user's current Performance Condition is +4, they can expect to perform outstandingly,
but at the same time, during the course of a workout, this number will decrease as fatigue from the exercise gets closer.\cite{performance-condition-firstbeat}\cite{performance-condition-garmin}

According to Firstbeat's support personnel, the formula behind the Performance Condition is protected information,
but generally, it uses a combination of personal background data, internal and external workload data, and special guidance from the Firstbeat analytics engine.\cite{firstbeat-performance-condition-emails}
Garmin's support website provides similar information: it is calculated based on the user's pace, heart rate and heart rate variability (HRV).\cite{performance-condition-garmin}

HRV is the variability in intervals between cardiac cycles.
It can be demonstrated, for example, by feeling one's pulse on the wrist while resting and breathing deeply - the interval shortens (heart beating faster) when breathing in, and lengthens (heart beating slower) when breathing out.\cite{hrv}
It's a complicated enough metric that to calculate it accurately, HRV should be measured using a chest strap,
an intrusive heart rate monitoring method which generally isn't available to a runner on a daily basis.

Additionally, the only activities that are currently supported for determining the Performance Condition, are running and cycling,
and the methods Firstbeat has developed to measure it has failed to satisfy a number of users, leading to them not paying much attention to the metric or outright ignoring the values.\cite{performance-condition-unreliable1}\cite{performance-condition-unreliable-reasoning}
One of the reasons behind their dissatisfaction is the fact that the metric only connects the effort the user is making with the distance they are covering,
 without taking into account the steepness of the slope, the humidity, temperature, and other external factors 
 and should be perceived more as an indicator of simply how much effort your body is making (as explained in BHerman's comment on a post\cite{performance-condition-unreliable-reasoning} in Firstbeat's forum), 
while being marketed as a highly precise metric that shows you how your training is going and not really explained well to the users.

\section{Heart rate zones}

Everybody has a resting heart rate (for example, right after waking up from a good night's sleep), and a maximum heart rate (the highest number of beats per minute achievable).
The range between HR min and HR max is commonly divided into five zones, based on the effort necessary and the effects the heart rate induces in a person:
\begin{itemize}
    \item Zone 1 (Warm up and very light exercise) --
    The trainee is relaxed at an easy pace and breathes rhytmically. Keeping one's heart rate in this zone helps with recovery from previous activities.
    \item Zone 2 (Easy, light exercise) --
    The trainee exercises at a comfortable pace, is able to hold a conversation and breathes slightly deeper. This zone is good for endurance practice.
    \item Zone 3 (Aerobic, moderate exercise) --
    At a moderate pace, it is more difficult to hold a conversation. Training in this zone helps improve aerobic capacity.
    \item Zone 4 (Threshold, hard exercise) --
    Training at a fast pace is somewhat uncomfortable, the trainee breathes forcefully. This zone is good for improving speed.
    \item Zone 5 (Maximum) --
    The trainee is at a sprinting pace, which is unsustainable for long period of time, and their breathing is laboured.
    This zone helps increase power in the trainee's body.\cite{garmin-heart-zones}\cite{polar-heart-zones}
\end{itemize}

Calculating HR min is an easy enough task - just measure your heart rate right after enjoying some good sleep; HR max is trickier.

\subsection*{Running methods to calculate HR max}
There are some methods to get HR max by running, such as running towards a hill at a high intensity and then ascending the hill at as high intensity as possible,
or, on flat ground, running 400 metres at high intensity and then increasing the intensity for another 400 metres.
These tests are considered stress tests, and individuals who are not highly active are not advised to take them. \cite{hrmax-running-tests}
Another method is based on interval running:
two intervals, four minutes each, in which the trainee is short of breath, interspersed by three minutes of active rest, and finished by a third interval, where two minutes in the trainee increases their speed as much as they can and run for as long as they can.
HR max is the maximum heart rate achieved.\cite{hrmax-running-tests-intervals}

\subsection*{Age-based methods to calculate HR max}
As the running methods are generally too difficult to do `on the spot', there have been attempts to calculate HR max based on readily available data.

The most commonly used formula to get one's maximal heart rate, $HRmax=220-age$ also known as the Fox formula, has been time and time again found largely inaccurate.
Originally intended as a rough formulation, with cross-sectional data from participants who are by no means representatives of a population\cite{220-hrmax-new-formula} and based on unoriginal research\cite{220-hrmax-disproved},
a number of researchers have tried to find a similarly simple but more accurate formulas, most of them claiming to have found the right one.
One of the popularly used is the cross-sectional research of Tanaka et al., who formulated the equation $208-0.7\times age$.
This formula was later validated by Gellish et al. in 2007 by conducting an analysis of longitudinal data from participants from various groups and developing a similar formula $207-0.7\times age$,
however, both Fox and Tanaka were disproved by Sarzynski et al. in 2014 as not precise enough based on standard error of estimate (12.4 and 11.4 bpm respectively) of the two formulas. \cite{hrmax-age-disproved}
In that study, Sarzynski stresses \textit{the importance of finding and validating other measures to be used in exercise prescriptions for the determination of intensity of exercise, the estimation of fitness levels, and as a criterion for achieving maximal exertion.}

The HUNT fitness study carried out on over three thousand people by Nes et al. found a more accurate formula $211-0.64\times age$, with the standard error of estimate of 10.8 bpm, but couldn't find evidence of interaction with gender, physical activity, VO2 max or BMI. \cite{hrmax-Nes-HUNT}


It seems that computational models for peoples' fitness leave a lot to be desired,
but this can all be remedied with further research.

\chapter{Existing solutions}
There's a plethora of applications taking advantage of wearable tech, not all of them using the available bio sensors, ranging from basic real-time heart rate monitoring to progress tracking, to calorie measuring, and to social network sized fitness communities.

In this chapter I will research a few such applications and judge them based on following criteria:
\begin{itemize}
    \item Use of IoT possibilities -- use of available sensors, data collection from the community,
    \item User fitness assessment -- how the user's fitness is assessed (attempt self-assessment, fill out a questionnaire, or take a physical test),
    \item Community -- support of interaction between users,
    \item Extra features -- interesting perks that make the platform stand out among others,
    \item User-friendliness -- navigation around the mobile application - finding the general settings, creating and clearing a route, general user experience (the applications were tested on an Android phone),
    \item Availability -- the platform's economical model and its suitability,
    \item Cross-platform -- major operating systems supporting the mobile application and sensors, if any are used,
    \item Propriety -- whether the solution is open-source, closed-source or else.
\end{itemize}

%==============================================================================================
\subsection{komoot}
The cross-platform application for outdoor track suggestion can also be used as a tour planner, a map and a navigation system.

\subsubsection*{Use of IoT possibilities}
Given its use of GPS sensors, komoot does qualify as a basic IoT system, however, it also relies heavily on user input for route rating.
The smart watch only gets used for displaying of routes and navigation, not taking any advantage of the available biosensors.
\subsubsection*{User fitness assessment}
When a user is creating a route, they can set its difficulty as one of five levels, 
describing the user's self-reported physical fitness and their current will to take up a challenge (or lack thereof): \textit{Couch Potato}, \textit{Average}, \textit{In Good Shape}, \textit{Athletic}, and \textit{Pro}.
This parameter is considered when the app is generating a suitable route -- presumably by trying to adjust the elevation profile of the possible routes.
The user's real fitness is not taken into account.

\subsubsection*{Availability}
The region-based pricing model allows users who do not travel too much to use the app for free,
since the first region is provided at no cost.
In the other pricing options Single Region, Region Bundle and All Regions -- the price-performance ratio seems to grow at a reasonable scale.
\subsubsection*{Community}
With features like sharing and recommending of routes users have taken, following of other users, upvoting and commenting on their posts, the mobile application integrates a full-fledged social network.
\subsubsection*{Extra features}
Komoot supports multiple sports, mainly hiking, running and multiple types of biking, to which it tailors the parameters of planned routes, such as mostly paved roads for road bikes and gravel for off-road biking.
There are other activities the app supported by the app, which, however, do not offer the customized planning and can only be tagged later.
These sports include skateboarding, skiing, climbing, and plenty of others.
This only serves as a tag for the user with no real impact on the app's functionality.

If a user decides to skip all planning and just go into the world, their route can be ``recorded" for their use or the use of other komoot users, who can pick this route and either

\begin{enumerate}[label=(\alph*)]
    \item follow it `as is', with all the possible deviations from the known network of streets and paths with the risk of impassable roads, or
    \item have komoot replan the off-grid segments and navigate through the known street network, or
    \item edit the route according to one's liking.
\end{enumerate}

Users can pick a starting point, a destination, as well as any number of waypoints in between for their route (see fig.\ref{komoot-nav-img})
Once the route is chosen, the user can go through the route's stats - the estimated time it will take to get from start to finish, its length, the elevation profile (see fig.\ref{komoot-route-details-img}, \ref{komoot-route-elevation-detail-img}) (uphill, downhill, highest and lowest points, estimated average speed) and the surfaces and their use in proportion to the route's length (see fig.\ref{komoot-route-surface-overview-img}).
All this information is delivered in easy-to-understand charts as well as interactive mappings -- using a slider, the user can see which stat applies to which part of the route.

\begin{figure}[h!]
    \centering
        \tmpframe{\includegraphics[width=0.4\textwidth]{Images/komoot-nav.png}}
        \caption{Route planning in the komoot app~\cite{komoot-nav-img}.}
        \label{komoot-nav-img}
\end{figure}

\begin{figure}[h!]
    \centering
        \tmpframe{\includegraphics[width=0.8\textwidth]{Images/komoot-route-details.png}}
        \caption{Route elevation profile~\cite{komoot-route-details-img}.}
        \label{komoot-route-details-img}
\end{figure}

\subsubsection*{User-friendliness}
The mobile app uses a bottom tab bar for navigation, however, the way the tabs are designed is not entirely user-friendly.
The `Plan' and `Record' tabs are clear enough, but the `Profile' tab contains a lot of information that does not seem entirely connected to the user's profile, such as people to follow and the app's settings.
The last tab, titled `More', only contains promotions for komoot offline maps and komoot premium, which is something a user does not expect under this tab.
The navigation through the mobile app takes some time to get used to -- it took me a while to find the Settings after I did not see them in the `More' menu on the bottom tab bar.

Once I created a route, the information provided was well-delivered and easy to read, however, there was no obvious way of completely cancelling the chosen route and picking another one.
Instead, hiding in the Options of the route -- which I had not noticed before -- I found the "Reset route" option, which cleared the route.
Overall, the app makers made some great UX decisions, but also some that do not seem natural to the user.
\subsubsection*{Cross-platform}
The system is fully integrated with the Apple Watch and Samsung gear, and -- at least limitedly -- supports a number of other brands of smart watches and other Bluetooth-enabled devices.
The mobile application runs both on iPhones and Android phones.
\subsubsection*{Propriety}
The system is not entirely open-source, as a few of their repositories are public, but the core components remain proprietary~\cite{komoot-github}.

\begin{figure}[h!]%
    \centering
        \subfig
        \tmpframe{\includegraphics[width=0.4\textwidth]{Images/komoot-route-elevation-detail.jpg}}
        \caption{Overview of the route's elevation~\cite{komoot-route-elevation-detail-img}.}
        \label{komoot-route-elevation-detail-img}
\end{figure}
\begin{figure}
    \centering
        \tmpframe{\includegraphics[width=0.4\textwidth]{Images/komoot-route-surface-detail.jpg}}
        \caption{Overview of the route's surface~\cite{komoot-route-surface-overview-img}.}
        \label{komoot-route-surface-overview-img}
\end{figure}

%==============================================================================================
\pagebreak
\subsection{endomondo}

Under Armour's endomondo application serves as a personal trainer for distance-based sports, such as running, cycling, or hiking.
Their main focus is on the endorphin release people get from encouragement and motivation -- this is why the app gives coaching advice, allows the users to send each other peptalks, and has over the course of its existence built a global fitness community.
\subsubsection*{Use of IoT possibilities}
The application integrates with a wide range of watches and sensors, collecting data from GPS trackers, bicycle speed and cadence sensors, bike power sensors and heart rate measurement units.
Some basic statistics based on this data (such as average heart rate) is presented to the user with a free plan, 
while the more interesting statistics displayed with interactive graphs are for premium users only -- average heart rate history, heart rate zone analysis, even weather information collected from AccuWeather.com at the time of the workout.

The application used to have a feature called Peer Banchmark, which used other users' data to compare the user with the rest of the community, however, this feature was removed due to lack of use~\cite{endomondo-HR-max-emails}.
\subsubsection*{User fitness assessment}
The premium version of the application promises personal exercise plans for running goals, and the heart rate zones are calculated using the Karvonen formula: $((max HR − resting HR) × \% Intensity) + resting HR$,
however, endomondo's website does not mention how and when HRmax and HRmin are calculated, and the official support personnel does not have any additional information about what is written on their website~\cite{endomondo-HR-max-emails}.
In spite of that, the values for the Karvonen formula are pre-filled when a user accesses the page, so they probably use age-based formulas to estimate all the variables.
In any case, they are all editable, so if the user already knows their HRmax for example from a laboratory test, they can use it.
When creating a new training plan (that is, having endomondo calculate the ``ideal'' plan for how a person should train if they want to run a specific distance on a specific day in the future),
one of the steps has the user choose a recent workout which will be used for calculating the individual's fitness level and for the plan generation~\cite{endomondo-training-plan-fitness-assessment}.
\subsubsection*{Availability}
Endomondo comes in two versions -- free and premium.
The free version has basic functionality, which is so limited that some have opted for free alternatives with more useful features such as Strava or Runmeter~\cite{endomondo-review}.
It is supported by not too obtrusive ads on the bottom of the screen.

The premium version is ad-free, brings a number of necessary features such as heart rate zones, training plans, and others, and is based on a monthly subscription.
\subsubsection*{Community}
Endomondo's main focus in keeping their users on the platform is motivation, which is also demonstrated by the application's slogan -- ``Free your endorphines''.
One of their features is a real-time audio motivator, which makes comments on how well the user is doing compared to previous workouts,
encouraging them if they are lagging behind or is informing them about the personal record they are be about to hit.

Competing against oneself, however, is not as much fun -- and therefore as motivating -- as getting to beat a friend or relative at the activity of choice.
This is why the application also integrates a social network of millions of users with functions like sharing one's activities, sending peptalks and challenges.
\subsubsection*{Extra features}
After the end of the workout, you can answer a ``How was your workout?'', however, since this is a plain text option, this data is only for the user's later reading, and cannot be used in any machine processing.

A user can post photos in the description of the activity, as well as use tags which are useful on social media after sharing.
With premium, one can filter based on these tags -- useful for tracking the number of kilometres that the user has run in a specific pair of shoes, or biked with a specific chain.
Other users can also be tagged, for example running partners or the person you are trying to beat in a challenge.

The premium version of the app also includes a few modes for interval training (periods of exercise alternating with periods of normal-to-high intensity training) as well as the option to build your own.
There are many applications that support this feature -- after all, it is just fairly basic time tracking -- but it is nice to have all the functionality in a single app.
\subsubsection*{User-friendliness}
The navigation in the mobile application is realized with a bottom tab bar, whose design is nicely done.
The `More' tab hides a lot of functionality, effectively serving as an easy-to-reach hamburger menu.

Given the fact that that endomondo does not support planning of routes and merely follows where the user is headed, the `Workout' tab which contains a map is less crowded than in a navigation app (fig.\ref{endomondo-home-img}).

\begin{figure}[h!]
    \centering
        \tmpframe{\includegraphics[width=0.5\textwidth]{Images/endomondo_home.jpg}}
        \caption{endomondo's main screen~\cite{endomondo-home-img}.}
        \label{endomondo-home-img}
\end{figure}

The main settings can immediately be found in the top-right corner of the main screen, or under the `More' tab, lost in between ten other options (fig.\ref{endomondo-more-tab}).
The application is easy to navigate and is fully made in, if somewhat blocky-feeling, material design.


\begin{figure}[htb!]
    \centering
        \tmpframe{\includegraphics[width=0.4\textwidth]{Images/endomondo-more-tab.jpg}}
        \caption{endomondo's `More' tab, with main settings lost in other options~\cite{endomondo-more-tab}.}
        \label{endomondo-more-tab}
\end{figure}

\begin{figure}[htb!]
    \centering
        \tmpframe{\includegraphics[width=0.4\textwidth]{Images/endomondo-bike-stats.png}}
        \caption{Statistics of a biking trip on endomondo~\cite{endomondo-bike-stats-img}.}
\end{figure}

\begin{figure}[htb!]
    \centering
        \tmpframe{\includegraphics[width=\textwidth]{Images/endomondo-history-example.png}}
        \caption{endomondo history example from the web application~\cite{endomondo-history-img}.}
\end{figure}

\subsubsection*{Cross-platform}
All the features are available for iOS 9 and newer and Android 4.1 and newer.
Endomondo also used to be available on Blackberry and Windows phones.
Some features are only available in the web application.
\subsubsection*{Propriety}
The whole endomondo application only has a single public repository on github, which is a fork of tapiriik -- a tool for data synchronization between different fitness applications, such as Endomondo, Garmin Connect, Strava, Runkeeper, and others~\cite{endomondo-tapiriik}.
Otherwise, the whole application is completely proprietary.
%==============================================================================================
\pagebreak
\pagebreak
\subsection{Strava}
Strava is a social network for athletes that offers a large number of features mainly focused on tracking and analysis of users' activities, sharing, and competing~\cite{strava}.
The target sports of the platform are running and cycling, but it supports plenty of other activities.
It has both a web and a mobile application, with some features shared and some only available in one of them.
\subsubsection*{Use of IoT possibilities}
Strava makes good use of sensors in the users' devices -- those for measuring heart rate, power, and GPS.
The platform creates heatmaps of routes that are used the most, taking advantage of their tracking technology and the users' GPS sensors.
They also use a novel way of determining the surface of routes, thanks to data not only from OpenStreetMap, but also from the users' bicycle frames.

\subsubsection*{User fitness assessment}
The Strava platform includes plenty of user fitness-related components, such as the premium `Fitness \& Freshness' feature,
which shows the user how fit they are on a graph of past activities, and how fresh, based on the amount of exercise the user has recently done,
all of this calculated using power data and heart rate or perceived exertion~\cite{strava-fitness-freshness}.
The platform also has heart rate zones, which are primarily calculated using an age formula, but the user can manually customize them.
\subsubsection*{Availability}
Strava is available both for free and for an affordable monthly subscription.
\subsubsection*{Community}
The platform has millions of active users who are connected in a large social network.
It allows users to create clubs, such as groups of friends as well as racing teams.
These clubs give the user the opportunity to keep track of the members' activity and exercise, with feeds and leaderboards.

For increased safety while outside, Strava Beacon is a premium feature that allows a user's friend to see their real-time location in case of an emergency.

Another community-building perk is the feed of challenges (fig.\ref{strava-challenges}), which help motivate the users with leaderboards and competitiveness.

\begin{figure}[h]
    \centering
    \tmpframe{\includegraphics[width=0.5\textwidth]{Images/strava-challenges.jpg}}
    \caption{Challenges feed on Strava~\cite{strava-challenges}.}
    \label{strava-challenges}
\end{figure}

\subsubsection*{Extra features}
One of the most popular premium features is called Segments (fig.\ref{strava-segments-img}), where users can compare their times on specific portions of road or trail using leaderboards.
This is supported by heatmaps of use frequency of specific routes.
It also allows the user to input their pictures, gear, and perceived exertion to the activity session.
The premium version of the platform offers a lot of features that are somewhat less connected to fitness but make it successful as a social network.

\begin{figure}[htb!]
    \centering
        \tmpframe{\includegraphics[width=0.5\textwidth]{Images/strava-segments.jpg}}
        \caption{Strava segments near the user's location~\cite{strava-segments-img}.}
        \label{strava-segments-img}
\end{figure}

\subsubsection*{User-friendliness}
The mobile app uses a tab bar for navigation, with nicely thought out tabs - `Feed' tab for posts from the community, `Explore' tab for challenges, clubs, and segments,
the central `Record' tab for the actual activity, which is simple and clutter-free, with the options to change the sport, pick a route, and turn on the Strava Beacon.
Another tab `Profile' keeps the user's account details and fitness information, such as their stats, activities, and training log.
The last tab `Training' contains the user's training plans, which are part of the premium subscription.

One thing that I missed was the option to create a route on the phone -- this can only be done in the web app and then synchronized with the mobile app, making the mobile application less suitable for ad hoc navigation.

Another unusual design decision was putting the `Find friends' button only on some pages.

\begin{figure}[h]
    \centering
    \tmpframe{\includegraphics[width=0.5\textwidth]{Images/strava-morning-run-stats.jpg}}
    \caption{Statistics of a run on Strava~\cite{strava-run-stats-img}.}
    \label{strava-run-stats-img}
\end{figure}

\subsubsection*{Cross-platform}
The mobile app Strava is available both on Android and iOS, as well as providing a web application.
It supports a variety of smart watches, power sensors, and other devices.
\subsubsection*{Propriety}
The application is closed-source, but it offers a publicly available API for custom development.
%==============================================================================================
\subsection{Garmin Explore and Garmin Connect}
The vast ecosystem Garmin has created for their users is filled with an assortment of wearables, radars, smart lights, navigations, customized maps, and plenty more, providing features which are presented to the user via modular mobile apps.

Here I will analyse the Garmin Explore application, which is targeted at hikers, in tandem with Garmin Connect, which functions as a fitness tracker.

\subsubsection*{Use of IoT possibilities}
Most fitness-focused Garmin watches do have a heart rate sensor and all of them include a step counter, however, this data is never compared with that of other users, keeping focus on the user's activity history with no prediction.
The Garmin Connect app includes an ``Insights'' feature, which compares the user's simpler data, such as daily number of steps or hours of sleep, to that of all other users.
\subsubsection*{User fitness assessment}
Some of the heart-rate-monitor enabled watches allow a user to set the zones (see chapter on fitness assessment) in which they would like to keep their heart rate depending on the activity they choose to do.
The Connect app also allows users to manually input their VO2 max, but doesn't test the user on it.

As mentioned in the Fitness assessment chapter, they offer the Performace condition stat, which however, according to user feedback, isn't useful or accurate.

\begin{figure}[h]
    \centering
        \tmpframe{\includegraphics[width=0.5\textwidth]{Images/garmin-connect-myday-screen.png}}
        \caption{My day - Garmin Connect's home screen displays statistics of the user's recent activity~\cite{garmin-my-day-img}.}
        \label{garmin-my-day-img}
\end{figure}

\subsubsection*{Availability}
In general, Garmin's pricing model relies on the user buying one of their high-end smartwatches, whereupon the rest of the components (apps, basic maps, support, etc.) are free, with some exceptions, such as advanced maps. 
The company's smartwatches are considered premium quality, with a wide price range - target groups include \textit{potential Rolex buyers}~\cite{garmin-expensive} as well as ordinary users~\cite{garmin-watches-review}.
\subsubsection*{Community}
The Connect application handles the data from the point of view of a fitness tracker, with social features like groups, competitions, likes, comments, and badges of accomplishment~\cite{garmin-connect}.
\subsubsection*{Extra features}
The Explore application is very simple, mostly just containing the bare necessities such as the map, route planning, and a history of activities.
It doesn't even contain navigation options; it is hard for me as a user to understand what its use is.
It allows the user to group waypoints, tracks, routes and activities into collections for each trip.
The Connect application, as a fitness tracker, gives insights into the user's fitness, allows to track the lifespan of the user's gear, their sleep quality, water intake, menstrual cycle (as this can affect the user's stamina), and other factorss.
\subsubsection*{User-friendliness}
The modularity of Garmin's apps allows for cleaner, easier-to-navigate interfaces.

The Garmin Explore app uses a tab bar for navigation; and while the four tabs do not have any inscriptions, the icons manage to communicate fairly clearly what the user can find in them.
In the Map tab, it took me a while to find the button for creating a new route or waypoint (top-left corner in fig.\ref{fig:garmin-map}), since it does not stand out from the other UI components.
The actual track creation process feels natural, visual and light-weight, with fast waypoint creation and editing, however, I lacked the option to search for a place that I would want to use as a waypoint.
A large disappointment for me as a user was the fact that I did not get a planned route -- only the polyline defined by the waypoints, which seems completely useless for navigation purposes.
Once the route was saved and shown on the map, I could not figure out how to cancel it.

\begin{figure}[h]
    \centering
        \tmpframe{\includegraphics[width=0.5\textwidth]{Images/Garmin-map-tab.jpg}}
        \tmpframe{\includegraphics[width=0.5\textwidth]{Images/Garmin-route-create.jpg}}
        \caption{Garmin's map tab~\cite{garmin-map-tab-img} and route creation screen~\cite{garmin-route-create}.}
        \label{fig:garmin-map}
\end{figure}

Another interesting decision is the Library tab, which contains the user's past activities, saved waypoints, routes, and tracks.
The difference between the (otherwise synonymic) words ``route'' and ``track'' is that a route is the generated polyline, whereas a track is a set of breadcrumbs of a past hike.

The Connect app, as the main fitness app in the Garmin ecosystem, entails a larger number of features, resulting in a full tab bar as well as a hamburger menu, which I found a little confusing.

\subsubsection*{Cross-platform}
All of Garmin's applications can be installed both on Android and iPhone.
However, they can only be paired with watches inside Garmin's ecosystem, which is a drawback for users who already own a smartwatch.
\subsubsection*{Propriety}
Some of the Garmin applications are completely open-source, either due to the software that they are derived from, or simply to provide the general public with opportunity to customize their Garmin experience~\cite{garmin-open-source}~\cite{garmin-connect-github-repos}.
These apps are mostly concerned with map handling, navigation and sound codecs, but also include some bits of functionality like widgets, low power watch apps and others.

While some of this code may be used in the Explore or Connect applications, these repositories don't contain their core functionality.

\subsection{Summary}
All of the studied platforms are used by a wide community of users, are either mostly or completely closed-source, base their maps off the OpenStreetMap, and have their own ways (or lack thereof) of integrating IoT principles in their design.

Komoot is a great app for navigation; its flexible way of route planning gives the user a good idea of what they can expect on their trip,
and includes navigation.
However, the information provided is not tailored to the user's actual fitness level.

Endomondo tracks the user's progress and motivates them.
Since there is no route planning, it doesn't estimate the difficulty of outdoor tracks -- instead it serves as a personal trainer.

Strava is great at tracking and analysing data from previous activity, and at providing new ways for athletes to connect and motivate each other.
It gives the user a good insight into their data, and allows comparison between users on pre-defined segments.
But again, there are no estimations done or tests to be taken.

Garmin Connect is a great fitness tracker, but the Explore app is not useful for route planning or navigation.

None of the apps fully customize their assessment of the user's fitness to estimate how they will do on their future hikes, which is a core feature of my application.

There is an abundance of community features in all of these apps; it is reasonable to consider how many of them are enough to keep the users entertained and how many become overwhelming.


\chapter{Functionality}
\linebreak
Based on the analysis of existing solutions, I will now describe the most important features that should be provided by my software.

I decided to construe the user-focused features as one would in a real-life agile project,
as this is in my experience more straightforward and easier to alter when necessary than flooding the reader -- for example, a developer -- with a long list of elaborate use cases right away.
The discussions that take place before reaching a specific decision should be well documented -- brief enough and detailed enough at the same time to provide good value after a quick read.
If the reader has access to a good transcript of these discussions, it is less likely that they will misunderstand the requested functionality;
in the best-case scenario, if the reader is the potential developer, they are an active part of those discussions, so they have a substantial impact on the outcome, thus diminishing the `coding monkey' phenomenon.

The top-down approach of the agile methodology which is perhaps the most common and natural to grasp is creating epics, decomposing them into user stories,
and with good knowledge of the project's architecture and after due consideration, add technical subtasks in an issue tracker of choice.

Epics are complex, high-level characterizations of desired functionality, which usually take longer to deliver than a few weeks.
An epic is usually comprised of description and multiple user stories, which cover the extent of the epic in higher granularity.
Once all the stories are delivered, the whole epic is considered finished.
Epics are sometimes formulated using the same format as user stories.

User stories (US) are short, simple descriptions of a feature told from the perspective of the person who desires the new functionality, usually a user of the system.\cite{user-story-definition}
They are not system specifications or functional requirements.
Rather, they are the beginning of a conversation that can lead to such specifications or requirements.
However, they provide context and scope of the requested feature.
Often they are written in simple, non-technical language, as most systems don't have tech-savvy users.
The typical template for a user story is:
\textit{As a < type of user >, I want < some goal > so that < some reason >.}\cite{user-story-definition}
Another important component of a US are \textit{acceptance criteria}, which define the functionality which needs to be working for the story to be finished, often including not only the happy path, but a sad path as well.
In general, acceptance criteria should be testable, which is why they are often used as starting points by testers, to get a better idea of what has been done, and provide them with some inspiration.

The main focus of this chapter will be to document my internal analytical discussions.
I will always start with a high-level description of the requested feature formulated as an epic or a user story and elaborate on its details incrementally, summing all of this up in acceptance criteria.
Some stories will also contain hints toward completion for the potential developer, which will be based on research done in previous parts of this thesis.
Since a part of this thesis will focus on implementing a proof of concept, some of the most imperative stories will also be fully processed and finished.

Features
\begin{itemize}
    \item F -- User's mood after finishing of track
\end{itemize}
%================================================================================================================================
\subsection*{E01 -- Heart rate monitoring}
The application should monitor a user's heart rate unobtrusively and reliably.

The user needs a method for getting their data to the mobile application and then to the backend, do it comfortably, preferably without any need for direct user---computer interaction.

\subsection*{E01-US01 -- Choose and integrate a wearable}
As an APP user, I want my heart rate-enabled wearable to work with the APP, so that I don't have to buy a new one.

Based on the analysis of existing applications, the most often used type of wearable tech are smartwatches.
While most applications support other devices, such as chest straps, watches are the most convenient and wide-spread gadgets to be used by the general public.
There are a few quite popular smartwatch vendors, most of whom have proprietary protocols of communicating with the system for data processing.\todo{recheck}
At the time of writing, however, Garmin seems like the most popular option, in spite of being more than a little costly.

\textbf{Acceptance criteria:}
\begin{itemize}
    \item A popular kind of wearables can communicate with the APP in both directions.
\end{itemize}

It will be more difficult to integrate some devices.
Some vendors don't disclose their devices' API, or the API's documentation is only available for use exclusively by partners of the given vendor,
some don't get paired with the phone itself and communicate with a proprietary app using server-based pairing instead (such as the Xiaomi Mi Band 4 I initially intended to use) and there is no easy or straightforward way to get the data.
For example, getting data from the Mi Band 4 would require root access to the user's phone\cite{miband4-server-based}, which is just not an option for a normal user.
Therefore I recommend limiting the supported devices to those that do not use server-based pairing, as this would cover most of the currently available gadgets.

In addition to these restrictions, continuous heart rate monitoring causes an extreme drain of the watch's battery, which is fine for short activities (sprints, swimming, cardio workouts), but can be a considerable issue while hiking.
This is why I recommend supporting mostly devices which have a large capacity in this regard.

For my PoC in this thesis, I chose the XXXXXXXXX.\todo{add what and why}

\subsection*{E01-US02 -- Manage devices paired with the APP}
As an APP user, I want to pair and unpair my wearables with the APP, so that the APP can use the data of all devices that I currently own.

\textbf{Acceptance criteria:}
\begin{itemize}
    \item The user can pair any number of compatible wearables.
    \item The user can unpair any number of compatible wearables down to zero.
\end{itemize}

The app should provide the user with the list of the phone's paired devices, so that they can choose their fitness gadgets.

\subsection*{E01-US03 -- Activate a device}
As an APP user, I want to mark a wearable I'm using as active, so that only its data is relevant to the statistics.

The user should be able to activate a device they want to use for monitoring of current activity or activity in the near future.

\textbf{Acceptance criteria:}
\begin{itemize}
    \item The user can mark a device as active.
    \item When a device is active, the platform can receive data from it.
    \item The user can mark a device as inactive.
    \item Only one device at a time is active; if activating a new one, the old device must be deactivated.
\end{itemize}    
%================================================================================================================================
\subsection*{E02 -- Objective user fitness assessment}
As an APP user, I want my fitness to be assessed, so that I can get relevant estimations of track difficulty.

Having analyzed the conventional ways of user fitness assessment in a previous chapter, the application should allow a user to use one of the multiple ways to get their fitness level.
The outcome of the assessment should be the individual's VO2 max index and their heart rate zones -- so that if they need to maintain their heart rate in a specific range, they can cross-reference it with the exhaustion they perceive and learn to recognize when they are in the desired zone.

If the user has no physical impairments, they can have their HR max assessed by a simple age-based formula, such as the one developed by Nes et al. (see the chapter on fitness assessment).

Users with physical impairments could self-evaluate how much their condition affects their fitness, and based on this their HR max can be set by using one of the standard formulas and subtracting a smaller or larger number of beats per minute (further research needs to be done on how many beats this should be).
This self-assessed value should be corrected as more data is available about the user.

As far as objective assessment, I see plenty of potential for the use of neural networks.
The networks can learn from the data collected from a good number of users over a longer period, and then categorize the new users into fitness groups based on this data, which will then be used to predict the users' future hikes.

\subsection*{E02-US01 -- Implement support for VO2 max tests}
As an APP user, I want to be able to take a guided test of my fitness in the APP, so that I don't have to look for tests elsewhere.

\textbf{Acceptance criteria:}
\begin{itemize}
    \item The user can take at least one VO2 max test using the APP.
    \item During the test, the user gets directions on their smartwatch.
\end{itemize}

The APP should implement a guide to multiple tests of VO2 max, with instructions on what they should do, and signals to direct the user while taking the test.
The instructions should also contain a description of the signals the user will receive.
The initial instructions should be textual and possibly audible, so that the user understands the aim of the test before taking it.
During the test, a watch vibration should alert the user about an incoming signal which will be displayed on the watch.
This signal would contain directions like 'Turn in N seconds' where N is a countdown to 'Turn now!'.
When no signals are occupying the device's screen, there should be pep talks like 'Keep going!' and 'Great job!'.
At all times during the test, there should be the test status on the device's screen: remaining time, distance walked, and any other metric relevant to the test.

As it is the most accurate of the more feasible tests, the application should encourage the user to take the 6-Minute Walking Test, and let them take it using the app in both 15- and 30-metre-long variations.
In order to calculate VO2 max, the APP can use Mänttäri's formulas.

\subsection*{E02-US02 -- Implement support for HR max tests}
As an APP user, I want to be able to take a guided test to find my heart rate zones in the APP, so that I don't have to look for tests elsewhere and get as accurate heart rate zones as possible.

The implementation should be similar to VO2 max tests; if possible, both metrics should be measured by a single test to reduce the time a user has to spend setting up the APP.

\textbf{Acceptance criteria:}
\begin{itemize}
    \item The user can take at least one HR max test using the APP.
    \item During the test, the user gets directions on their smartwatch.
\end{itemize}%================================================================================================================================
\subsection*{E03 -- Manual input of biometric data}
The users might not want to take an exhausting test at the time of setting up the APP.
This is why they should be allowed to input the data in other ways.

If a public universal database with all people's medical data existed, it would make sense to get the necessary information directly via an API.
However, as no such thing exists (and for good reasons), we will let the user enter their biometric data manually or possibly by setting up an integration with a popularly used fitness app that does measure the necessary data.

\subsection*{E03-US01 -- Allow manual input of basic biometric values}
As an APP user, I want to provide the APP with my basic metrics, so that the results are relevant to me.

\textbf{Acceptance criteria:}
\begin{itemize}
    \item User can enter their basic metrics - height, weight, age, sex - manually.
\end{itemize}

If the user has a smart scale, it could be integrated to get the most recent values, however, that seems like a bit of an overkill, considering that it's just one number - the weight.

The APP needs the user's height, weight, biological sex and age.
The age should be calculated based on their date of birth, so that it is always reasonably accurate.
However, I myself have had issues with disclosing my exact date of birth to anybody, and the accuracy would be good enough with even just the month and year.
Yet again, it is reasonable to consider whether a user who doesn't have an issue giving out their heart rate and fitness data would show similar concern.

\subsection*{E03-US02 -- Allow manual input of advanced biometric values}
As an APP user who has recently done lab fitness tests, I want the APP to use these values instead of doing the APP's tests, so that I don't have to take fitness tests that are most probably less accurate than what I know.

A user should be able to manually enter the VO2 max, resting heart rate and maximum heart rate, as well as the severity of the effect a potential condition might have on the individual's fitness.
The APP should ask whether the user has such a condition and if there is a maximum heart rate (possibly recommended by the doctor) the user shouldn't exceed when active.

\textbf{Acceptance criteria:}
\begin{itemize}
    \item User can enter their extended metrics - VO2 man and HR max - manually.
    \item The APP asks about the severity of any conditions that the user has and its impact on their fitness.
    \item The APP asks about a recommended maximum heart rate the user shouldn't exceed.
\end{itemize}
%================================================================================================================================
\subsection*{E04 -- Tracks and maps}

\subsection*{E04-US01 -- Integrate a map}
As an APP user, I want to use a map to visualize my hikes, so that I can follow them easily.

\textbf{Acceptance criteria:}
\begin{itemize}
    \item An interactive map is displayed in the APP.
\end{itemize}

Since the APP is meant to be used as a hike planner, the user needs a simple to follow aid to help them plan their tracks.
The whole APP should be focused around the hikes a user takes and around their difficulty given the user's fitness level.

There aren't many different maps available for use by developers.
While Google Maps would probably be a popular choice, Google decided to put their maps behind a paywall\cite{google-maps-paywall} in 2018.
Therefore it makes sense to choose the open-source OpenStreetMap\cite{OpenStreetMap} and use one of its derivatives with a public API for route planning (such as OpenRouteService\cite{OpenRouteService}).

\subsection*{E04-US02 -- Choose a track}
As an APP user, I want to plan a route from point A to point B.

\textbf{Acceptance criteria:}
\begin{itemize}
    \item A user can pick a starting point from the map.
    \item The user's current location can be automatically found and set as the starting point.
    \item A user can choose points on the map through which they want to pass.
    \item A user can choose a destination point from the map.
    \item A user can remove points from the planned route.
    \item The points in a planned route can be rearranged in any order.
    \item If the last point is removed, the previous point becomes the last point.
    \item If multiple different tracks are available, the user can compare their attributes.
    \item If multiple different tracks are available, the user can choose one of them.
\end{itemize}

\subsection*{E04-US03 -- History of tracks hiked}
As an APP user, I want to see which routes I've taken while using the APP, so that I can compare my impressions of the tracks.

Every track is taken over a network of geographical coordinates, which means one can create thumbnails from its projection on the map.
This, along with the names of nearby prominent landmarks, such as the mountain range through which the user hiked, and the date of the hike, should be enough for the user to identify the track they were searching for.
Since there is a number of attributes a track can have, it should be possible to filter by them.
Filters should include the date on which the route was taken, the time it took to complete it, the route's length.
Since there can be hundreds of such routes, the list should be paginated.

\textbf{Acceptance criteria:}
\begin{itemize}
    \item Show a paginated, filterable list of routes taken by the user.
\end{itemize}

\subsection*{E04-US04 -- Favourite tracks}
As an APP user, I want to highlight and retake some planned routes, so that I can compare my results over time.

\textbf{Acceptance criteria:}
\begin{itemize}
    \item A user can plan a brand new track and set it as a favourite.
    \item A user can choose a previous track and set it as a favourite.
    \item A user can remove a track from favourites.
    \item User can name favourite tracks.
    \item User can choose a favourite track and hike it again.
\end{itemize}

\subsection*{E04-US05 -- Edit a copy of a track}
As an APP user, I want to create new, similar tracks, based on existing ones, so that I don't have to enter all the details again.

\textbf{Acceptance criteria:}
\begin{itemize}
    \item User can choose an existing track and edit its copy.
\end{itemize}

\subsection*{E04-US05 -- Display planned track details}
As an APP user, I want to see interesting information about a track that has been planned for me, so that I can make an informed decision whether or not to take it.

When tapping a track, its details should be shown: its outline on a map, length, terrain in its segments, its elevation profile, colours of each official hiking track it uses, and interesting places to see along the track.
This is also the place for information that is customized for the user's fitness -- such as how long it will probably take them to hike it, how difficult its segments will be for the user in terms of perceived exertion and expected heart rate zones.
This should take into account the limitations of the user's fitness, which they may have entered according to E03-US02.

\textbf{Acceptance criteria:}
\begin{itemize}
    \item User can interactively see basic details of a planned track, including the track on a map, its length, terrain analysis, elevation profile, as well as interesting places on the track.
    \item User can see customized biometric details of a planned track, including the amount of time the hike will take them, the relative difficulty of the track's segments, and expected heart rate zones.
    \item User can update the preferred heart rate zone ad hoc and the customized details should be recalculated.
\end{itemize}

\subsection*{E04-US06 -- Display old track details}
As an APP user, I want to see interesting information about a track I've hiked, so that I can check it in a few days and show my friends.

\textbf{Acceptance criteria:}
\begin{itemize}
    \item User can see all the basic info as on a planned track.
    \item User can see the originally estimated values of biometric indicators and the real values -- but only if they hiked the track that was originally planned for them. If they didn't follow it, the original estimations are no longer valid.
    \item User can see a graph of their heart rate during the hike.
\end{itemize}

%================================================================================================================================

\subsection*{E05 -- User profile}
The user needs a place to review and change their provided information, as well as a possibility to take the provided tests.

\subsection*{E05-US01 -- GDPR anonymization}
As an APP user, I want to be able to practice my right to erasure or anonymization of my data, so that I know my data is safe.

\textbf{Acceptance criteria:}
\begin{itemize}
    \item User can press a button to anonymize their data. To avoid mistakes, there should be a confirmation dialogue.
    \item Data anonymization runs after confirmation.
    \item User's old data cannot be connected to them.
\end{itemize}

\subsection*{E05-US02 -- Management of provided biometric information}
As an APP user, I want to be able to change the information I've entered into the APP, so that the values are always up to date and my estimates remain accurate.

\textbf{Acceptance criteria:}
\begin{itemize}
    \item User can add and delete weight entries.
    \item User can edit the height entry, the recommended maximum heart rate, and the severity of health conditions.
    \item User can edit the advanced biometric data.
\end{itemize}

\todo{more stories on user management}

%================================================================================================================================

\subsection*{E06 -- Hike with friends}

Since hiking is generally an activity for small groups of people, it makes sense to be able to create groups in which everybody will know the hike's details.

\subsection*{E06-US01 -- Adjust the calculations to friends' fitness}
As an APP user, I want the estimations to be accurate even if I'm hiking with friends.

In a diverse group of people, someone's fitness level will likely be lower than that of the others, which means that the whole hike will take longer than if this person wasn't in the group.
This is why the estimated time spent on the hike should be calculated based on the `weakest link', and each participant's expected heart rate zones and all other biometric estimations should be adjusted accordingly.

The estimations can be adjusted either to the weakest participant's capabilities, or the system can acknowledge the motivation that they will feel when with others and slightly overestimate.

It wouldn't be too user-friendly to expose the name of the weakest individual, as they might feel ostracised, disheartened and less likely to go on hikes with their friends for fear of holding them back (and, by extension, use the APP that made them feel that way).

\textbf{Acceptance criteria:}
\begin{itemize}
    \item 
\end{itemize}

%==========Invite friends to hikes======================================================================================

\subsection*{E06-US02 -- Invite friends to hikes}
As an APP user, I want to invite friends to my hikes, so that we all have access to the same track when planning it.

The invitation should include the proposed date of the hike and all the hike's details, including the time the hike will take if the user joins the group.

\textbf{Acceptance criteria:}
\begin{itemize}
    \item 
\end{itemize}

\subsection*{E06-US03 -- Refuse an invitation to hike with friends}
As an APP user, I want to refuse both new and accepted invitations to hikes that I don't want to attend, so that the hike organiser knows I won't come and my invitation list isn't cluttered.

\textbf{Acceptance criteria:}
\begin{itemize}
    \item 
\end{itemize}

\subsection*{E06-US04 -- Accept an invitation to hike with friends}
As an APP user, I want to accept invitations to hikes that I want to attend, so that I get updates and plan around the event.

\textbf{Acceptance criteria:}
\begin{itemize}
    \item 
\end{itemize}

\subsection*{E06-US05 -- Take back an invitation}
As an APP user, I want to take back an invitation I sent to users unwittingly, so that they don't get confused.

\textbf{Acceptance criteria:}
\begin{itemize}
    \item 
\end{itemize}

%=============Friends and friendship request management================================================

\subsection*{E06-US06 -- Add friends}
As an APP user, I want to add new friends, so that I can invite them to hikes.

The only people that can be added, need to be users of the system.
In order to add friends, the APP can use user profile links and social media; a unique way would be using the phones' proximity if the two people are close to one another, for example with Bluetooth or NFC.

\textbf{Acceptance criteria:}
\begin{itemize}
    \item User A can send a friendship request to user B via a link to user A's profile.
    \item User A can send a friendship request to user B via popular social media.
    \item User A can send a friendship request to user B via Bluetooth or NFC.
    \item If the request is accepted, user A can invite user B to hikes and vice versa.
\end{itemize}

\subsection*{E06-US07 -- Remove friends}
As an APP user, I want to remove people I no longer meet with from friends, so that they can't invite me to hikes anymore.

\textbf{Acceptance criteria:}
\begin{itemize}
    \item If user A and user B are friends, both can remove the other user from their list of friends.
    \item If user B is removed, neither user A or user B can see each other's data or invite each other to hikes.
    \item If user B is removed, the hikes (and related data) user A and user B took together remain in their respective histories.
\end{itemize}

\subsection*{E06-US08 -- Take back a friendship request}
As an APP user, I want to take back a friendship request I sent to users unwittingly, so that they don't get confused.

\textbf{Acceptance criteria:}
\begin{itemize}
    \item If user A sends a friendship request to user B and user B doesn't accept it in the meantime, user A can take it back.
    \item If user A sends a friendship request to user B and user B accepts it, the request cannot be taken back.
\end{itemize}

\subsection*{E06-US09 -- Accept a friendship request}
As an APP user, I want to accept a friendship request that someone sent me, so that we can invite each other on hikes.

\textbf{Acceptance criteria:}
\begin{itemize}
    \item When user A sends a friendship request to user B, user B can accept the request.
    \item When user B accepts a friendship request from user A, they can both invite each other for hikes.
    \item When users are friends, they cannot send new friendship requests to each other.
\end{itemize}


\subsection*{E06-US10 -- Dismiss a friendship request}
As an APP user, I want to dismiss a friendship request when I don't know the sender or don't want to hike with them, so that my request list isn't cluttered.

\textbf{Acceptance criteria:}
\begin{itemize}
    \item When user A sends a friendship request to user B, user B can dismiss the request.
    \item When user B dismisses a friendship request from user A, neither of them can invite the other one for hikes.
    \item When user A sends a friendship request to user B, user B cannot send a friendship request to user A; they can only accept or dismiss the existing request.
\end{itemize}


%=======Peek at track statistics================================================================================================

\subsection*{E06-US09 -- Peek at track statistics without inviting friends}
As an experienced hiker and APP user, I want to see the statistics of a difficult track if I were to invite a specific group of people,
so that I can decide whether it makes sense to invite them or not, since a very unfit person might not be able to get through or will hold up the whole group more than is acceptable.

\textbf{Acceptance criteria:}
\begin{itemize}
    \item User A can make a group of their friends, choose a track and the system calculates 1) the track's attributes based on the weakest user, and 2) user A's resulting biometrics.
\end{itemize}
\todo{this could be a premium feature}

\subsection*{E06-US10 -- Manage people in group}
As an APP user, I want to add and remove people from my potential hiking group 

\subsection*{E06-US11 -- Send hike invites after peeking}
As an APP user, I want to send invites to all the friends in the group that I created in order to peek at our common track statistics, so that I don't have to do it manually for each one.


%=========Considered features================================================================================================

\subsection*{Considered features}
One feature that could help make the platform into a social network is messaging.
I decided against it because everybody is already using their messaging apps of choice to communicate and it would be distracting the user from planning their hikes.



\chapter{Mobile app UI design}
Bassed on the desired functionality of the platform described in the previous chapter, I was able to create a design for the user interface -- UI of the mobile application.

In the beginning of this UI design project, I first looked at what would be the platform the project should run on.
Since it is a mobile application, the range of options was slimmed down to two -- either Android or iOS,
given their prevalence on the market (Android on the first place with 70.68\% of the market share, iOS second with 28.79\% as of April 2020)~\cite{market-share-mobile-os}.
It is partially due to Android's crushing lead on iOS devices, but also because of my personal experience with the more popular OS, that I decided to design the app for Android.

One of the most important decisions to make when designing an application is the navigation in it, and from there there's only a small step to information architecture (IA) -- the \textit{science of organizing and structuring content of the [\dots] mobile applications}~\cite{information-architecture}.
The IA of an application helps the designer decide which concepts in the application are high-level and discrete enough
to be used as `umbrella terms' for navigating to other, deeper concepts, while feeling intuitive to the user.

However, when reading up on design guidelines for either of the systems, I realised that Android's instructions seem outdated.
Backed by plenty of discussions online~\cite{hamburger-discoverabillity} as well as my own user experience with the native applications, the typical hamburger menu ☰ has gone out of style.

One of the reasons is the increasing size of handheld devices.
A hamburger menu (or `navigation drawer') located in the top left corner is nearly inaccessible for the majority of users, who are right-handed and use their thumb to operate the device~\cite{thumb-zone-article}.

\begin{figure}[h!]
    \centering
    \tmpframe{\includegraphics[width=\textwidth]{Images/thumb-zone.png}}\hfill
    \caption{Thumb-zone mapping for left- and right-handed users~\cite{thumb-zone-img}}
    \label{fig:sign-up-email}
\end{figure}

Another reason is the fact that the hamburger menu decreases discoverability of features, as they are hidden by default~\cite{hamburger-discoverabillity}.

Lastly, using the hamburger menu often tempts the designer to just put everything there, without giving too much consideration to the use of the features.

One of the most popular alternatives is the tab bar -- or `bottom navigation' in Android documentation.
This type of navigation is mainly implemented by Apple's designers, but is fast spreading to Android-powered devices.
It's a bar on the bottom of the screen containing three to five labelled icons representing \textit{tabs} where each leads to a different location in the app.
It enforces the idea that a user should always know where in the application they are, what they can do there and where they can go because it's always on the screen (except when the keyboard is being used).
Another one of its advantages is the possibility of inobtrusive notifications using icon badges -- little red circles in the top right corner of the icon, often containing a number to show the number of notifications.
Other navigation options include dropdown menus, scrollable menus, single page dot navigation, and others~\cite{hamburger-alternatives}.

When describing the screens I will refer to the user stories which the discussed UI component covers.

I chose the tab bar with five tabs: `Plans', `Routes', `Map', `Friends', and `Profile'.
However, when installing the APP, the user needs to set the APP up.

\section{Set up screens}
The welcome screen contains all the options for signing up, with a button to switch to the sign-in screen if the user already has an account.
Signing up with e-mail creates a new account for the user based on their e-mail~(fig.\ref{fig:sign-up-email}), while using a social network account takes the user to a sign-in screen of the specific social network.
Similarly, the user can sign in to their existing account~(fig.\ref{fig:sign-in-email}).

\begin{figure}[h!]
    \centering
    \tmpframe{\includegraphics[width=0.4\textwidth]{Images/appScreens/signUp1.pdf}}\hfill
    \tmpframe{\includegraphics[width=0.4\textwidth]{Images/appScreens/signUp2.pdf}}
    \caption{Welcome screen with sign-up options and e-mail sign-up screen}
    \label{fig:sign-up-email}
\end{figure}

\begin{figure}[h!]
    \centering
    \tmpframe{\includegraphics[width=0.4\textwidth]{Images/appScreens/signIn1.pdf}}\hfill
    \tmpframe{\includegraphics[width=0.4\textwidth]{Images/appScreens/signIn2.pdf}}
    \caption{Welcome screen with sign-in options and e-mail sign-in screen}
    \label{fig:sign-in-email}
\end{figure}

After signing up and creating a new account, there is a sequence of screens asking for the user's personal data, such as height, weight, biological sex, birth date, and whether they know about a condition that means their heart rate should be limited, and if so, how much (on a scale of 1 to 5).
Another screen asks the user to pair a watch with the APP, showing a list of Bluetooth devices paired with the phone.
The last one asks the user to take a fitness test, so that their fitness can be assessed.
This step can be skipped, but there's a warning about the inaccuracy of methods based on readily available data compared to taking a fitness test.
At the same time, the user is assured that the test can be taken later.

If the user decides to take the test, they are rerouted to the screen with fitness tests (without the bottom navigation bar).
Again, it is possible to quit this screen.

\section{Profile tab}
After setup, the APP opens on the `Profile' tab, displaying all entered and calculated information on cards, as well as the option to log out, and the currently active device~(fig.\ref{fig:profile1}).

\begin{figure}[h!]
    \centering
    \tmpframe{\includegraphics[width=0.4\textwidth]{Images/appScreens/profile1.pdf}}\hfill
    \caption{The Profile screen}
    \label{fig:profile1}
\end{figure}

The card with fitness opens to show the user's fitness data -- VO2 max, heart rate zones, and minimum and maximum heart rate.
The heart rate zones are color-coded; these colors occur in the whole application to signify difficulty.
There's an icon to get more information about the metrics (excerpts from the Fitness Assessment chapter), and the option to take a fitness test~(fig.\ref{fig:fitnessInfo}).
The pictures are placeholders for charts of development of the metric over time.

\begin{figure}[h!]
    \centering
    \tmpframe{\includegraphics[width=0.4\textwidth]{Images/appScreens/profile-fitnessInfo.pdf}}\hfill
    \caption{Fitness information}
    \label{fig:fitnessInfo}
\end{figure}

On the `Fitness tests' screen, the user can choose between multiple tests, each of which has a tag saying what it tests~(fig.\ref{fig:fitnessTests}).
As an example, the `6 Minute Walking Test' screen has directions on what the user should do, and after the test, there is a screen with results~(fig.\ref{fig:6mwt}).
Again, the pictures will contain the charts of the metric's values over the course of the test.

\begin{figure}[h!]
    \centering
    \tmpframe{\includegraphics[width=0.4\textwidth]{Images/appScreens/profile-fitnessTests.pdf}}\hfill
    \caption{Fitness tests}
    \label{fig:fitnessTests}
\end{figure}

\begin{figure}[h!]
    \centering
    \tmpframe{\includegraphics[width=0.4\textwidth]{Images/appScreens/profile-fitness-6mwt.pdf}}\hfill
    \tmpframe{\includegraphics[width=0.4\textwidth]{Images/appScreens/profile-fitness-6mwt-results.pdf}}
    \caption{6 Minute Walking Test}
    \label{fig:6mwt}
\end{figure}

When we come back to the `Profile' screen (fig.\ref{fig:profile1}), we can also manually edit all the information the user has provided.
The profile picture, basic data like height etc., as well as the fitness information.

Another option from here is to see the devices by expanding the card with the active device.
The `Devices' screen~(fig.\ref{fig:devices}) contains the active device which can be deactivated, as well as other paired devices, which can be either activated or unpaired.
The user can also pair a new device with the app (opening a list of devices paired with the phone).
Tapping the `Activate' button will deactivate the currently active device.

\begin{figure}[h!]
    \centering
    \tmpframe{\includegraphics[width=0.4\textwidth]{Images/appScreens/profile-devices.pdf}}\hfill
    \caption{Devices}
    \label{fig:devices}
\end{figure}

\section{Map tab}
The default tab on which the application opens is the `Map' tab~(fig.\ref{fig:map}).
This is where the user can look at the integrated map, as well as search for specific places~(fig.\ref{fig:map-searchPlace}) and add them as waypoints to the planned route.
The waypoint will be added to the route as the `next' waypoint.

\begin{figure}[h!]
    \centering
    \tmpframe{\includegraphics[width=0.4\textwidth]{Images/appScreens/maps1.pdf}}\hfill
    \caption{Map screen}
    \label{fig:map}
\end{figure}

\begin{figure}[h!]
    \centering
    \tmpframe{\includegraphics[width=0.3\textwidth]{Images/appScreens/map-search1.pdf}}\hfill
    \tmpframe{\includegraphics[width=0.3\textwidth]{Images/appScreens/map-search2.pdf}}\hfill
    \tmpframe{\includegraphics[width=0.3\textwidth]{Images/appScreens/map-search3.pdf}}\hfill
    \caption{Search for a place}
    \label{fig:map-searchPlace}
\end{figure}

Planning a new route can also be started with the floating button on the bottom right of the screen~(fig.\ref{fig:map}).

The planning process consists of three steps:
\begin{enumerate}
    \item waypoints,
    \item pick a route, and
    \item plan a hike.
\end{enumerate}

In the `Waypoints' step~fig.\ref{fig:plan-waypoints}), the user picks where they want to start their route, the destination, and all the points through which they want to pass.
Each of the points can be picked directly from the map with a long press, or by searching for a place~(fig.\ref{fig:map-searchPlace}).
The current location can also be used as a starting point.
The intermediate points can be added using the plus button, and all the waypoints can be rearranged by drag'n'dropping each line by the menu icon on the left.
The user can also remove individual waypoints using the X icon on the right: when removing the starting or destination point, only the text is removed, since every route needs two points.

Another option is to choose a return trip, where the platform should offer routes that start in the starting point, go through the destination, and either give a different route to go back if possible, or has the user retrace their steps.

\begin{figure}[h!]
    \centering
    \tmpframe{\includegraphics[width=0.4\textwidth]{Images/appScreens/map-plan1.pdf}}\hfill
    \tmpframe{\includegraphics[width=0.4\textwidth]{Images/appScreens/map-plan2.pdf}}\hfill
    \caption{Planning a route with waypoints}
    \label{fig:plan-waypoints}
\end{figure}

When the user is satisfied with the waypoints, they tap the `Find a route' button.
In this step the planning UI pulls up higher, covering a larger part of the map, while the visible part of the map contains all the route options on an appropriately zoomed-in map.
The chosen option is highlighted, while the others are less opaque.
The options can be switched between by tapping on the map.

The middle of the planning UI now contains the details of the chosen route -- the time it will get the user to hike it if they go alone, its elevation, length, prevalent surface type, and a difficulty chart.

The bottom part of the planning UI has information on individual segments of the route.
The picture is a placeholder for the route's elevation profile, which has two interactive sliders to pick the segment whose details the user wants to know.
These details include its elevation, surface, and a difficulty chart.

The route can be saved and unsaved by tapping the bookmark icon, and shared via link by tapping the share icon.

If the user wants to just start the hike without planning for a different day or inviting friends, they can press the `GO' button and the APP navigates him,
or they can continue to the next step: Plan a hike.

\begin{figure}[h!]
    \centering
    \tmpframe{\includegraphics[width=0.4\textwidth]{Images/appScreens/map-plan3.pdf}}\hfill
    \tmpframe{\includegraphics[width=0.4\textwidth]{Images/appScreens/map-plan4.pdf}}\hfill
    \caption{Picking a route based on its details.}
    \label{fig:plan-pick-route}
\end{figure}

The last step is planning a hike.
This means choosing the date with a date picker, and the people to join the user~(fig.\ref{fig:plan-plan-hike}).
These can be searched for among the user's friends, using the `invite people' button~(fig.\ref{fig:plan-invite-friends}).
While searching for friends, their avatars are added to the summary on the bottom and the counter is incremented.

\begin{figure}[h!]
    \centering
    \tmpframe{\includegraphics[width=0.4\textwidth]{Images/appScreens/map-plan5.pdf}}\hfill
    \caption{The initial state of a planned hike.}
    \label{fig:plan-plan-hike}
\end{figure}

\begin{figure}[h!]
    \centering
    \tmpframe{\includegraphics[width=0.4\textwidth]{Images/appScreens/map-plan6-friends.pdf}}\hfill
    \tmpframe{\includegraphics[width=0.4\textwidth]{Images/appScreens/map-plan7-friends.pdf}}\hfill
    \caption{Searching for friends to invite to a hike. A summary of already chosen people is on the bottom.}
    \label{fig:plan-invite-friends}
\end{figure}

Once the friends are chosen, the user can see a summary of the hike, with the difficulty chart adjusted to the group's common fitness (fig.\ref{fig:plan-end}), the number of people who are coming or invited, and how long it is going to take them.

This screen also contains a checkbox to allow friends join the hike -- whether the hike should be marked as joinable when it is shared to the friends' feed of `All plans'~(fig.\ref{fig:plan-all-plans}).

Once the planning is finished, we return to the basic `Map' screen, with a snack bar notification telling the user where they can find the planned hike and how many people were invited~(fig.\ref{fig:plan-end}).

The whole planning can be stopped at any point with the `Cancel' button, which will trigger a confirmation dialogue, as the user is about to delete all their progress, and after confirming, we return to the `Map' tab.

\begin{figure}[h!]
    \centering
    \tmpframe{\includegraphics[width=0.4\textwidth]{Images/appScreens/map-plan8.pdf}}\hfill
    \tmpframe{\includegraphics[width=0.4\textwidth]{Images/appScreens/map-plan9.pdf}}\hfill
    \caption{Overview of the hike. After pressing the `Finish' button, we are back to the `Map' screen.}
    \label{fig:plan-end}
\end{figure}

\section{Plans tab}
The `Plans' tab is divided into three sections -- `All plans', `My plans' and `History'.
The default `All plans' section~(fig.\ref{fig:plan-all-plans}) is a filterable feed of the users' and their friends' upcoming plans.
This is also where the user's invitations to friends' hikes are displayed; these can be either dismissed or accepted.

A collapsed hike card contains basic information, the route on a map (represented by the picture placeholder) such as who hosts it, when it is happening and how long it will take, whether it is joinable and how difficult it would be for the user.

When tapping on the card, the card expands to show more of the hike's details: 
who is coming, how long it is currently planned to be plus how much longer it will take if the user joins, and all of its information recalculated to the user's fitness.

If the hike is joinable, there is an `Ask to join' button; if not, the user can only go back.

Again, there are the options to save the route, and share either the route or the hike, which is chosen in a dropdown dialogue.

\begin{figure}[h!]
    \centering
    \tmpframe{\includegraphics[width=0.4\textwidth]{Images/appScreens/plans-all-plans.pdf}}\hfill
    \tmpframe{\includegraphics[width=0.4\textwidth]{Images/appScreens/plans-all-plans-joinable-detail.pdf}}\hfill
    \caption{All plans feed. Detail of a joinable hike.}
    \label{fig:plan-all-plans}
\end{figure}

The `My plans' section~(fig.\ref{fig:plan-my-plans}) contains the hikes that the user has planned or joined.
The detail of a hike planned by the user is different, compared to looking at hikes planned by other people,
since there have to be options to administer the hike.
Requests to join are shown in this detail; by tapping at the `People' section the user can manage the people connected to the hike, as well as allow or disallow people to join the hike.
The route can be cancelled by the red trash icon, and the user can also let others know that they can't go.
If the user is the host of the hike, they have to appoint a new host from the people who have accepted the invitation.

The `Start' button will become active on the day of the hike.
When it is pressed, the route will be activated on the phones of all the attendants, allowing to switch to navigation mode.

Tapping on the `Route' section will show the route's details, just like in fig.\ref{fig:plan-pick-route}.

\begin{figure}[h!]
    \centering
    \tmpframe{\includegraphics[width=0.3\textwidth]{Images/appScreens/plans-my-plans.pdf}}\hfill
    \tmpframe{\includegraphics[width=0.3\textwidth]{Images/appScreens/plans-my-plans-detail.pdf}}\hfill
    \tmpframe{\includegraphics[width=0.3\textwidth]{Images/appScreens/plans-my-plans-detail-people.pdf}}\hfill
    \caption{My plans. Detail of a hike planned by the user, and a view of the invited and going people.}
    \label{fig:plan-my-plans}
\end{figure}

The `history' section~(fig.\ref{fig:plan-history}) of the `Plans' tab is the same as the `My plans' section, except all the hikes are in the past.
Each hikes' map contains the planned route, and the real route in case the user didn't precisely follow the plan.
This hike can be repeated by pressing the `Plan a hike' button and choosing whether to use the originally planned or the actually taken route.
Then the user is again taken to the planning UI.

\begin{figure}[h!]
    \centering
    \tmpframe{\includegraphics[width=0.4\textwidth]{Images/appScreens/plans-history-detail.pdf}}\hfill
    \caption{Detail of a historic hike.}
    \label{fig:plan-history}
\end{figure}

\section{Routes tab}
The `Routes' tab~(fig.\ref{fig:routes}) contains a filterable list of saved routes.
The only difference between a hike card and a route card is the fact that routes are not planned, so they don't include a host, people who are going, or the date they are planned for.

\begin{figure}[h!]
    \centering
    \tmpframe{\includegraphics[width=0.4\textwidth]{Images/appScreens/routes.pdf}}\hfill
    \tmpframe{\includegraphics[width=0.4\textwidth]{Images/appScreens/routes-detail.pdf}}\hfill
    \caption{Routes tab. Detail of a route.}
    \label{fig:routes}
\end{figure}

\section{Friends tab}
Finally, the last tab is the `Friends' tab.
It contains a view of all the user's friends, as well as any pending friendship requests.
The detail of a user's profile is similar to the user who is signed in, but contains only basic information.
The profile is shareable, and the friend can also be removed from the user's friends list.
This is confirmed by a confirmation dialogue, warning the user about the fact that they will be removed from one another's plans.

\begin{figure}[h!]
    \centering
    \tmpframe{\includegraphics[width=0.3\textwidth]{Images/appScreens/friends.pdf}}\hfill
    \tmpframe{\includegraphics[width=0.3\textwidth]{Images/appScreens/friends-find.pdf}}\hfill
    \tmpframe{\includegraphics[width=0.3\textwidth]{Images/appScreens/friends-detail.pdf}}
    \caption{Friends tab, finding new friends and the user profile of a friend.}
    \label{fig:routes}
\end{figure}

\section{Navigation mode}
The application allows a user to navigate a planned route using a map and written directions (fig.\ref{fig:navigation}).
Once a hike is started, the view is switched to the `Map' tab.
The route is visualized on the map in colours of respective hiking trails and the user's position is highlighted.
There's also an overview of the difficulty of the current part of the route, and the user is notified if they are lagging behind the plan.
If the user wants to see something else than the navigation mode, they can use the close button on the top right, or just switch between tabs.
The navigation mode stays alive and accessible through a permanent notification.

\begin{figure}[h!]
    \centering
    \tmpframe{\includegraphics[width=0.4\textwidth]{Images/appScreens/navigation.pdf}}
    \label{fig:navigation}
\end{figure}
\chapter{Architecture}
In this chapter I will describe the high-level architecture of the IoT platform, based on the component diagram.

Internet of Things, as a system, requires outlying units ``on site'', which measure the data and send it to the processing centers.
These units are in my case both the smartwatches and mobile phones.

The smartwatch requires an application to propagate the data from the heart rate sensor to the mobile application through the \code{hr\_data} interface.
Once the raw data is received, it is combined with GPS data from the phone and forwarded through the \code{data\_access} interface to the IoT platform, where it gets processed and saved.
When the smartphone application requires it, it requests the processed data through the same interface and displays it.

\begin{figure}[h]
    \tmpframe{\includegraphics[width=\textwidth]{Images/component.pdf}}
    \caption{Component diagram of the designed IoT solution.}
\end{figure}

Depending on the number and complexity of the features in the future, it is also possible to introduce a web client in order to make the smartphone APP lighter and easier to navigate.

The watch communicates with the phone application over Bluetooth, with the phone application acting as a server, waiting for the watch application to connect and send its data.
The phone application connects to the server using the phone's Internet connection.

\section{Smartwatch application}
Since the vendors of smartwatches generally use proprietary operating systems, every integrated smartwatch type needs an application of its own.
This is distributed together with the phone application.

Generally, it should be as light-weight as possible, to spare the limited battery resources.
Without any need for persistence, the application layer should collect the data from the heart rate sensors and regularly send them to the phone in batches.
The presentation layer displays data from the phone (such as instructions for a fitness test or navigation directions) received through the \code{directions} interface.
This is why it requires a bi-directional channel for Bluetooth communication.

\section{Smartphone application}
The phone application needs to be able to communicate with the watch via Bluetooth using the \code{hr\_data} and \code{directions} interfaces, as well as with the APIs for data access, notification and  user authentication offered by the server.

The user is authenticated via \code{user\_auth}.

The \code{Database} also contains the user's plans, so that the user does not have to be online when in navigation mode.

When navigating, the \code{Mobile\_app} collects data from the available \linebreak\code{GPS\_sensor} and from the \code{Watch\_app}, and tries to send it to the server via the \code{data\_access} interface if there is an active Internet connection.
Otherwise this is stored in the local \code{Database} and sent when the phone next connects.

The \code{Mobile\_app} is also subscribed to a messaging service such as Firebase~\cite{firebase} provided by the server to handle notifications.

\section{Server application}
The \code{IoT\_platform} application communicates with the \code{Mobile\_app} via the aforementioned interfaces, and also requires a public interface from every supported social network, so that the users can log in using their accounts from those networks and add new friends that they already know.

The server application has a simple front-end client for administration purposes.

The \code{IoT\_platform} is hosted in a cloud, since this infrastructure provides room for scalability and performance~\cite{cloud},
which should work perfectly in combination with my application's need for storing of detailed route definitions and computing tailored route difficulty estimates.

The offered \code{user\_auth} interface is used for authentication.
The user's data should be as secure as possible given its very personal nature, while maintaining a comfortable level of usability for the user;
this is why authentication should be done with OAuth, as opposed to HTTP Basic Auth, combined with a One Time Password for logins from unrecognized devices.

\chapter{PoC implementation}
In this chapter I describe the process of designing and implementing the PoC solution.

\section{PoC implementation}

I started by implementing a simple Android application that acts as a server for incoming Bluetooth connections from paired devices.
This was fairly straightforward since Android's documentation is up to date and there is a large community of developers for this OS.

The next step was to choose and integrate a smartwatch.

Initially I meant to use the Xiaomi Mi Band 4 since I already owned one.
However, Xiaomi watches do not pair directly with the phone, but with a proprietary application using server-based pairing,
and there is no way to access the data from the phone other than to reverse engineer the protocol, root the phone, and access the pairing keys in the app's database~\cite{miband4-server-based}.

I looked at multiple vendors' popularity based on market share and affordability, their documentation to see how good their development support is and whether or not they have public APIs for development.

According to Statista.com~\ref{fig:watch_market_share}, the most popular are Apple watches, followed by Samsung, Fitbit and other vendors.

\begin{figure}[h]
    \tmpframe{\includegraphics[width=\textwidth]{smartwatch_market_share.png}}
    \caption{Market share of smartwatch unit shipments worldwide from summer 2014 to spring 2020, by vendor~\cite{watch_market_share}.}
    \label{fig:watch_market_share}
\end{figure}

While Apple has a crushing lead on all other vendors, buying a unit just for development is not an option for me because of their high prices.
This is why I chose the Samsung Galaxy Watch Active.

Samsung watches run Tizen OS, which has public APIs, extensive documentation, tutorials and a large community.
This is why I was surprised to find out how difficult it is to develop for this watch.
The watch needs to be connected over the local network to the Tizen Studio IDE;
the tutorial for this~\cite{tizen_connect_tutorial} omits the important step of enabling developer options when working with Bluetooth,
so that the device does not disconnect when trying to use the Bluetooth API.
The tutorial also doesn't mention that Bluetooth should not only be disabled, but has to be turned off, on, and off again in order to connect.

Working with Bluetooth itself was tricky as well since their guide~\cite{tizen_bluetooth_guide} explicitly says that it should be implemented in the main thread,
but the app refused to connect to the phone when it was implemented that way.
When I finally found a working sample application~\cite{tizen_bluetooth_sample}, its Bluetooth functionality was divided into different threads.

When implementing the watch's communication with the Android phone, another surprise was the missing step of asking for permissions to access the heart rate sensor~\cite{tizen_sensor_tutorial}.

When my application could finally both measure heart rate and send string data to the Android phone, I attempted to send the heart rate data,
which the application successfully performed and then crashed, without an accessible crash log, stack trace, or any log that could shed light on the issue at hand.
After spending a large amount of time on trying to find a solution to this problem without any results, even trying to run my app on a different watch (Samsung Galaxy Watch Active 2), I figured out that the problem was bigger than I initially expected and decided to leave the implementation unfinished.

\section{Recommendations for future development}
When it comes to wearables, I recommend limiting the supported devices to those that do not use server-based pairing, as this would cover most of the currently available gadgets,
or take advantage of (and contribute to) existing open-source software for wearables integration, such as Gadgetbridge~\cite{Gadgetbridge}.

In addition to these restrictions, continuous heart rate monitoring causes an extreme drain of the watch's battery,
which is fine for short activities (sprints, swimming, cardio workouts), but can be a considerable issue while hiking.
This is why I recommend supporting mostly devices which have a large capacity in this regard.


\setsecnumdepth{part}
\chapter{Conclusion}
\linebreak
For now, I have been spending most of the time thinking about the importance of different features, evaluating the costs of having some compared to others, and how the use cases of existing apps apply to mine.

I have studied the various ways one can assess a person's fitness based on heart rate and decided to use VO2 max in my application.

I have also thought out a high-level architectural model for communication between the components.

\bibliographystyle{iso690}
\bibliography{mybibliographyfile}

\setsecnumdepth{all}
\appendix

\chapter{Acronyms}
\begin{description}
	\item[IoT] Internet of Things -- \textit{The interconnection via the Internet of computing devices embedded in everyday objects, enabling them to send and receive data} \cite{IoT-dictionary} without requiring human-to-human or human-to-computer interaction. \cite{IoT-definition-no-interaction}
	\item[PoC] Proof of Concept -- \textit{Evidence, typically deriving from an experiment or pilot project, which demonstrates that a design concept, business proposal, etc. is feasible.} \cite{PoC-dictionary}
\end{description}


\chapter{Contents of enclosed CD}

change appropriately

\begin{figure}
	\dirtree{%
		.1 readme.txt\DTcomment{the file with CD contents description}.
		.1 exe\DTcomment{the directory with executables}.
		.1 src\DTcomment{the directory of source codes}.
		.2 tizen-app\DTcomment{smartwatch app sources}.
        .2 android-app\DTcomment{smartphone app sources}.
        .2 thesis\DTcomment{the directory of \LaTeX{} source codes of the thesis}.
		.1 text\DTcomment{the thesis text directory}.
		.2 thesis.pdf\DTcomment{the thesis text in PDF format}.
	}
\end{figure}
\end{document}

% arara: xelatex
% arara: xelatex
% arara: xelatex


% options:
% thesis=B bachelor's thesis
% thesis=M master's thesis
% czech thesis in Czech language
% english thesis in English language
% hidelinks remove colour boxes around hyperlinks

\documentclass[thesis=B,english]{FITthesis}[2019/03/21]

\usepackage{todonotes}

%\usepackage[utf8]{inputenc} % LaTeX source encoded as UTF-8
% \usepackage[latin2]{inputenc} % LaTeX source encoded as ISO-8859-2
% \usepackage[cp1250]{inputenc} % LaTeX source encoded as Windows-1250

% \usepackage{subfig} %subfigures
% \usepackage{amsmath} %advanced maths
% \usepackage{amssymb} %additional math symbols

\usepackage{dirtree} %directory tree visualisation

% % list of acronyms
% \usepackage[acronym,nonumberlist,toc,numberedsection=autolabel]{glossaries}
% \iflanguage{czech}{\renewcommand*{\acronymname}{Seznam pou{\v z}it{\' y}ch zkratek}}{}
% \makeglossaries

% % % % % % % % % % % % % % % % % % % % % % % % % % % % % % 
% EDIT THIS
% % % % % % % % % % % % % % % % % % % % % % % % % % % % % % 

\department{Department of Software Engineering}
\title{Thesis title (SPECIFY)}
\authorGN{Zuzana} %author's given name/names
\authorFN{Václaviková} %author's surname
\author{Zuzana Václaviková} %author's name without academic degrees
\authorWithDegrees{Zuzana Václaviková} %author's name with academic degrees
\supervisor{Jan Krákora}
\acknowledgements{THANKS (remove entirely in case you do not with to thank anyone)}
\abstractEN{Summarize the contents and contribution of your work in a few sentences in English language.}
\abstractCS{V n{\v e}kolika v{\v e}t{\' a}ch shr{\v n}te obsah a p{\v r}{\' i}nos t{\' e}to pr{\' a}ce v {\v c}esk{\' e}m jazyce.}
\placeForDeclarationOfAuthenticity{Prague}
\keywordsCS{Replace with comma-separated list of keywords in Czech.}
\keywordsEN{Replace with comma-separated list of keywords in English.}
\declarationOfAuthenticityOption{1} %select as appropriate, according to the desired license (integer 1-6)
% \website{http://site.example/thesis} %optional thesis URL


\begin{document}

% \newacronym{CVUT}{{\v C}VUT}{{\v C}esk{\' e} vysok{\' e} u{\v c}en{\' i} technick{\' e} v Praze}
% \newacronym{FIT}{FIT}{Fakulta informa{\v c}n{\' i}ch technologi{\' i}}

REMOVE THIS
\linebreak
Analysis of an IoT solution for assessment of tourist track physical difficulty

The aim of this thesis is to analyse, design and implement a proof-of-concept \todo{maybe change the way the acronyms are introduced?} (PoC) Internet-of-Things (IoT) solution for tourist track difficulty estimation.
Using a mobile application, the user shall be able to select a track and assess how difficult the specific part of the track will be for them, based on data collected from similarly fit users.
\begin{itemize}
    \item Analyse and compare similar existing solutions.
    \item Analyse and design an IoT solution for data processing.
    \item Design a mobile application for data visualization and track selection.
    \item Create a PoC of the IoT platform and of the mobile application.
    \item Collect raw data from smart watch heart rate sensors and smart phone GPS locators and propose relevant statistics to process this data.
\end{itemize}


\setsecnumdepth{part}
\chapter{Dictionary}
\begin{description}
	\item[IoT] Internet of Things -- \textit{The interconnection via the Internet of computing devices embedded in everyday objects, enabling them to send and receive data} \cite{IoT-dictionary} without requiring human-to-human or human-to-computer interaction. \cite{IoT-definition-no-interaction}
	\item[PoC] Proof of Concept -- \textit{Evidence, typically deriving from an experiment or pilot project, which demonstrates that a design concept, business proposal, etc. is feasible.} \cite{PoC-dictionary}
\end{description}

\chapter{Introduction}
With the increasing popularity of smart wearables in the general public, a growing number of people have been able to benefit from adjustments in their behaviour according to the way their data gets processed and fed back to them.
Runners can barely imagine not checking how fast and how efficiently they just ran their daily track or how much they have improved over the last month,
people with sedentary jobs attempt to keep their daily step count above a currently recommended number,
masses of people have been saving up for the Apple Watch 5~\cite{AppleWatch5} in spite of the ECG health monitor being virtually unnecessary for those not at risk of cardiac disease~\cite{ecg-screening}, etc.

All of this data is collected from a variety of devices, such as smartwatches, textiles with integrated sensors, jewelry, or more invasive ones such as implants or tattoos, all of which are meant to be worn on the user's body and which measure and send data for analysis - hence the name \textit{wearable technology}, or \textit{smartwear}~\cite{what-is-wearable-tech}.

Thanks to the data collected from these devices, the users have been keeping track of their health and fitness, setting goals for themselves and tracking personal achievements.
However, all of this is related to the past of each individual user, while the true potential of smartwear lies in collecting data from thousands of people,
finding patterns and correlations in large datasets and using the results to increase precision and improve upon whatever the data is being used for by the virtue of prediction.

That being said, the result of this thesis will be a design for an IoT solution, mainly to be used by hikers who want to not only pick a track on a map and see how long it is going to take them to get from point A to point B (which generally isn't tailored to the individual's fitness and often doesn't account for the possible rapid changes in the track's elevation profile),
but also an objective difficulty level of particular sections of said track, which may have additional benefits for people with cardiac issues or those trying to lose weight, who are interested in some more detailed insight into the exertion induced by the track in order to keep their heart rate in a specific range.
All of this will be available to the user before stepping foot on said track, thanks to the data collected from users with various fitness levels and their own history,
while the user's fitness will be presented more accurately as they continue to use the application.

I will study how fitness is assessed in regard to feasibility, availability for the general public, and precision, in order to provide fundamentals for further work.

Using methods of software engineering, I am going to analyse the use of wearable technology in existing applications and document the most common features as well as their outstanding traits and their use of IoT principles (or lack thereof).
Based on this analysis, I will pick the most relevant features to include in my design and extend them with features of my own devising, which will set my solution apart from existing software.

I will design and implement a PoC IoT solution -- a mobile application which will collect data from the smartwear sensor and an available GPS locator, along with a backend IoT platform to receive and process collected data, and the communication channels used between these two modules.
I will neither design nor implement a full-fledged IoT platform as that's not in the scope of this thesis.
I will design (not implement) the frontend mobile application for displaying of processed data, track selection with integrated maps and explore the necessary legal requirements for the use of such maps.

I will examine the use of different device types with heart rate sensors and choose one with which I will build my PoC.

I will collect and perform suitable operations on the collected data in order to visualize it.



\setsecnumdepth{all}
\chapter{Existing solutions}
\todo[color = green]{
what is IoT

compare with just fit watches and apps - those only have past data from you.
Mine has statistically important data from others - how is mine original comapred to other apps
compare with app Strava

IoT data security - GDPR
data anonymization

human fitness groups, self-assessment

different smart watches
Specific smartwatch protocol, what format does it use

how to get the data out of the smartwatch
how to accept the data into the android app
how to send the data to the IoT platform (REST, GSM, power line communication?)
geofence technology or others (geohash)

data science - how to statistics
maps integration
}

\todo{split into logical parts and order them so they make sense chronologically}

\todo{try swot analysis of my idea, minimalistic or large app? opportunities that other apps do/don't have, fill a market hole
maybe a chapter on economics - app as a startup}

\chapter{Analysis and design}

\chapter{Realisation}

\setsecnumdepth{part}
\chapter{Conclusion}


\bibliographystyle{iso690}
\bibliography{mybibliographyfile}

\setsecnumdepth{all}
\appendix

\chapter{Acronyms}
% \printglossaries
\begin{description}
	\item[GUI] Graphical user interface
	\item[XML] Extensible markup language
\end{description}


\chapter{Contents of enclosed CD}

%change appropriately

\begin{figure}
	\dirtree{%
		.1 readme.txt\DTcomment{the file with CD contents description}.
		.1 exe\DTcomment{the directory with executables}.
		.1 src\DTcomment{the directory of source codes}.
		.2 wbdcm\DTcomment{implementation sources}.
		.2 thesis\DTcomment{the directory of \LaTeX{} source codes of the thesis}.
		.1 text\DTcomment{the thesis text directory}.
		.2 thesis.pdf\DTcomment{the thesis text in PDF format}.
		.2 thesis.ps\DTcomment{the thesis text in PS format}.
	}
\end{figure}

\end{document}

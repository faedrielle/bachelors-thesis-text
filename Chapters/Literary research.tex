\todo[color = green]{
what is IoT

compare with just fit watches and apps - those only have past data from you.
Mine has statistically important data from others - how is mine original comapred to other apps
compare with app Strava

IoT data security - GDPR
data anonymization

human fitness groups, self-assessment

different smart watches
Specific smartwatch protocol, what format does it use

how to get the data out of the smartwatch
how to accept the data into the android app
how to send the data to the IoT platform (REST, GSM, power line communication?)
geofence technology or others (geohash)

data science - how to statistics
maps integration
}

\todo{split into logical parts and order them so they make sense chronologically}

\todo{try swot analysis of my idea, minimalistic or large app? opportunities that other apps do/don't have, fill a market hole
maybe a chapter on economics - app as a startup}

\subsection{Existing solutions analysis}
There's a plethora of applications taking advantage of wearable tech, not all of them using the available bio sensors from basic real-time heart rate monitoring to progress tracking, to calorie measuring, and to social network sized fitness communities.

In this chapter I will research a couple \todo[color = green]{reword} such applications, compare and rate them based on some chosen criteria. \todo[color = green]{based on what? list the criteria and describe each point}
\begin{itemize}
    \item availability - free/microtransactions/paid
    \item cross-platform - major operating systems supporting the mobile application and sensors, if any \todo[color = green]{reword}
    \item use of IoT possibilities -- data collection from the community, big data
    \item use of available sensors
    \item user-friendliness
    \item similarity in features
    \item user fitness assessment -- self-assessment, questionnaire
    \item extra features
    \item open source
\end{itemize}


\subsection{komoot}
The cross-platform application for outdoor track suggestion can also be used as a tour planner, guide and a navigation system.

While being rich with features, komoot doesn't base its functionality on objective data - it's the users themselves who rate and recommend the specific 'routes' (as they are named here),
which in its nature \todo[color = green]{wording} is prone to error and there are only limited ways of eliminating the human factor.

Moreover, since the app is not open source, \todo[color = green]{finish sentence}

\begin{itemize}
    \item Availability --
\end{itemize}

\subsection{endomondo}

\subsection{Strava}

\subsection{myFitnessPal}
\linebreak
With the increasing popularity of smart wearables in the general public, a growing number of people have been able to benefit from adjustments in their behaviour according to the way their data gets processed and fed back to them.
Runners can barely imagine not checking how fast and how efficiently they just ran their daily track or how much they have improved over the last month,
people with sedentary jobs attempt to keep their daily step count above a currently recommended number,
masses of people have been saving up for the Apple Watch 5 \cite{AppleWatch5} in spite of the ECG health monitor being virtually unnecessary for those not at risk of cardiac disease \cite{ecg-screening}, etc.

All of this data is collected from a variety of devices, such as smart watches, textiles with integrated sensors, jewelry, or more invasive ones such as implants or tattoos, all of which are meant to be worn on the user's body and which measure and send data for analysis - hence the name \textit{wearable technology}, or \textit{smartwear}. \cite{what-is-wearable-tech}

Thanks to the data collected from these devices, the users have been keeping track of their own health and fitness, setting goals for themselves and tracking personal achievements.
However, all of this is related to the past of each individual user, while the true potential of smartwear lies in collecting data from thousands of people,
finding patterns and correlations in huge datasets and using the results to increase precision and improve upon whatever the data is being used for by the virtue of prediction.

That being said, the result of this thesis will be a design for an IoT solution, mainly to be used by hikers who want to not only pick a track on a map and see how long it is going to take them to get from point A to point B (which generally isn't tailored to the individual's fitness and often doesn't account for the possible rapid changes in the track's elevation profile),
but also an objective difficulty level of particular sections of said track, which may have additional benefits for people with blood pressure issues, who are interested in a more detailed insight into the heart rate induced by the track\todo[color=green]{wording}.
All of this will be available to the user before stepping foot on said track, thanks to the data collected from users with various fitness levels and their own history,
while the user's fitness will be presented more accurately as they continue to use the application.

Using methods of software engineering, I am going to analyse the use of wearable technology in existing applications and document the most common features as well as their outstanding traits and their use of IoT principles (or lack thereof).
Based on this analysis, I will pick the most relevant features to include in my design and extend them with features of my own devising, which will set my solution apart from existing software.\todo[color=green]{is this sentence necessary?}

I will design and implement a PoC IoT solution -- a mobile application which will collect data from the smartwear sensor and an available GPS locator, along with a backend IoT platform to receive and process collected data, and the communication channels used between these two modules.
I will neither design nor implement a full-fledged IoT platform -- that's not in the scope of this thesis.
I will design (not implement) the frontend mobile application for displaying of processed data, track selection with integrated maps and explore the necessary legal requirements for the use of such maps.

I will examine the use of different device types with heart rate sensors and choose one with which I will build my PoC.

I will also perform a data analysis and propose suitable operations to be executed on the collected data, which will be included in the processing of this data on the IoT platform.

The aim of this thesis was to analyse and design the functionality of a PoC IoT solution for assessment of tourist track difficulty, based on the individual's fitness.

To support my research with knowledge in the fitness domain, I studied the modern methods of cardiovascular fitness assessment, judged them based on their feasibility, accuracy and relevance to my platform.
This gave me a good insight into measurable attributes of fitness and what is required of my application for it to be user-friendly when it came to fitness testing while remaining accurate.

Thanks to this research I could evaluate the suitability of fitness estimation models used by existing applications in the domain of fitness and e-health.
I also studied their use of IoT principles and opportunities, their community features, user-friendliness of their mobile applications, the supported platforms and propriety.
With this inspiration, I saw which features are crucial for the applications in the domain, and how I can make mine stand out.
One of the most important such features is the option to use similar users' data to predict the exertion of a specific user.

With this in mind, I took to figuring out the functionality that the platform ought to support, using the approach of agile methodology.
The result of this effort is a chapter of epics and user stories, ready to be handed over to an agile development team.
A number of features was suggested for future consideration.

All these requirements were manifested in the front-end design for an Android mobile application, which is also ready to be implemented and tested with users, and greatly exceeded the original aims of this thesis.

Another step was designing a high-level architecture for the IoT platform, and finally, the development of a PoC.
Development efforts were hindered by technical issues of the chosen smartwatch's operation system, misguiding documentation (or lack thereof) and absence of examples for this particular problem.

Due to these issues I was not able to collect the data which I needed to demonstrate the PoC.

More research needs to be done, in coordination with data scientists and physical therapy professionals, in order to develop reliable models for fast difficulty assessment of a route for a specific individual.